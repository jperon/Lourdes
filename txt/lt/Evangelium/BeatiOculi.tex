\x~Sequéntia sancti Evangélii secúndum Lucam.

In illo témpore : Dixit Iesus discípulis suis : Beáti óculi, qui vident quæ vos videtis. Dico enim vobis, quod multi prophétæ et reges voluérunt vidére quæ vos videtis, et non vidérunt : et audire quæ audítis, et non audiérunt. Et ecce, quidam legisperítus surréxit, tentans illum, et dicens : Magister, quid faciéndo vitam ætérnam possidébo ? At ille dixit ad eum : In lege quid scriptum est ? quómodo legis ? Ille respóndens, dixit : Díliges Dóminum, Deum tuum, ex toto corde tuo, et ex tota ánima tua, et ex ómnibus víribus tuis ; et ex omni mente tua : et próximum tuum sicut teípsum. Dixítque illi : Recte respondísti : hoc fac, et vives. Ille autem volens iustificáre seípsum, dixit ad Iesum : Et quis est meus próximus ? Suscípiens autem Iesus, dixit : Homo quidam descendébat ab Ierúsalem in Iéricho, et íncidit in latrónes, qui étiam despoliavérunt eum : et plagis impósitis abiérunt, semivívo relícto. Accidit autem, ut sacerdos quidam descénderet eádem via : et viso illo præterívit. Simíliter et levíta, cum esset secus locum et vidéret eum, pertránsiit. Samaritánus autem quidam iter fáciens, venit secus eum : et videns eum, misericórdia motus est. Et apprópians, alligávit vulnera eius, infúndens óleum et vinum : et impónens illum in iuméntum suum, duxit in stábulum, et curam eius egit. Et áltera die prótulit duos denários et dedit stabulário, et ait : Curam illíus habe : et quodcúmque supererogáveris, ego cum redíero, reddam tibi. Quis horum trium vidétur tibi próximus fuísse illi, qui íncidit in latrónes ? At lle dixit : Qui fecit misericórdiam in illum. Et ait illi Iesus : Vade, et tu fac simíliter.
