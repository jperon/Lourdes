% Afficher des recommendations concernant la syntaxe.
\RequirePackage[orthodox,l2tabu]{nag}
% Paramètres du document.
\documentclass[%
a5paper%                       Taille de page.
,11pt%                         Taille de police.
,DIV=15%                       Plus grand => des marges plus petites.
,titlepage=on%                 Faut-il une page de titre ?
,headings=optiontoheadandtoc%  Effet des paramètres optionnels de section.
,headings=small%
,parskip=false%
,openany%
]{scrbook}
\renewcommand*\partheademptypage{\thispagestyle{empty}}

%\usepackage{geometry}

% Par souci de clarté, la définition des commandes est reportée dans un document annexe.
\usepackage{gredoc,mudoc,lyluatex}
\usepackage{pdfpages,transparent,array,ltablex}
\addtolength{\voffset}{2mm}\addtolength{\headsep}{-2mm}
\setlength{\extrarowheight}{2mm}
\addto\captionsfrench{%
  \renewcommand{\indexname}{Index des chants}%
}

\pdfcompresslevel=9

\newcommand{\lieu}[1]{\hfill\linebreak[3]\hspace*{\stretch{1}}\nolinebreak\mbox{\emph{(#1)}}}

\newcommand{\commandement}[1]{\noindent\textbf{#1}}

\newcommand{\schola}[1]{}\newcommand{\foule}[1]{#1}
\providecommand{\dest}{foule}

\newcommand{\bgimage}[1]{%
\raisebox{-.45\paperheight}[0pt][0pt]{%
  \transparent{0.3}%
  \includegraphics[width=.7\paperwidth,height=.7\paperheight,keepaspectratio=true]{img/#1}%
  }%
}

\def\arraystretch{1.2}



\title{Pèlerinage du Christ-Roi}
\date{}

\let\oldaddchap\addchap
\def\addchap#1{\oldaddchap{#1}\markright{Pèlerinage du Christ-Roi}}

\def\blindsection#1{\markright{#1}\addcontentsline{toc}{section}{#1}}

\begin{document}

%\maketitle
{\pagestyle{empty}
\foule{\includepdf{Couverture}}
\schola{\includepdf{Couverture-schola}}
\clearpage}


{\def\arraystretch{1}
{\centering\Large\textbf{Programme du pèlerinage}\par}

\medskip\thispagestyle{empty}


\begin{tabularx}{\textwidth-\parindent}{l!{:}X}
\multicolumn{2}{l}{\textbf{Samedi 25 octobre}}\\
14h00	& Chapelet. \lieu{Basilique Saint-Pie X}\\
14h30	& Messe de l'Immaculée Conception.
	  \lieu{Basilique Saint-Pie X}\\
16h00	& Chemin de Croix. \lieu{Basilique Saint-Pie X}\\
18h00	& Conférence\footnote{Les conférences seront données en français, italien, espagnol, anglais et allemand.}.
	  \lieu{Église Sainte-Bernadette}\\
21h15	& Procession aux flambeaux devant la Grotte.\\
\multicolumn{2}{l}{\centering\textbf{Dimanche 26 octobre}}\\
9h00	& Chapelet.
	  \lieu{Basilique Saint-Pie X}\\
9h30	& Messe du Christ-Roi.
	  \lieu{Basilique Saint-Pie X}\\
15h30	& II\iemes\ Vêpres, procession et bénédiction des malades.
	  \lieu{Basilique Saint-Pie X}\\
18h00	& Conférence.
	  \lieu{Église Sainte-Bernadette}\\
20h30	& Exposition du Saint-Sacrement − Nuit d'adoration.
	  \lieu{Basilique Saint-Pie X}\\
\multicolumn{2}{l}{\textbf{Lundi 27 octobre}}\\
9h00	& Chapelet.
	\lieu{Basilique Saint-Pie X}\\
9h30	& Messe de saint Pie X.
	\lieu{Basilique Saint-Pie X}\\
11h15	& Chapelet médité à la Grotte.\\
12h15	& Clôture du pèlerinage.\\
\end{tabularx}

}

\clearpage


\hspace*{\stretch{1}}

{\centering\LARGE\bfseries\thispagestyle{empty}
Centenaire

de la mort de Saint-Pie X

\smallskip
1914 − 2014

\vspace*{\stretch{1}}

\Huge
Livret du Pèlerin

\medskip\Large
25 − 26 − 27 octobre 2014

\vspace*{\stretch{2}}

\normalsize
Fraternité Sacerdotale Saint-Pie X\par
}


\cleardoublepage


\addchap{Ordinaire de la Messe}

\vspace{-.7\baselineskip}
\subsection*{Prières au bas de l'autel}\blindsection{Messe des catéchumènes}
\vspace{-.3\baselineskip}

\rubrica{Le prêtre et les ministres qui l'assistent se préparent à la célébration de la Liturgie par les prières dites au bas de l'autel. À la messe chantée elles sont dites à voix basse ; on se reporte directement à l'Introït.}

\versio{%
In nómine Patris, et Fílii, et Spíritus Sancti. Amen.}{%
Au nom du Père, et du Fils, et du Saint-Esprit. Ainsi soit-il.}

\versio{%
℣. Introíbo ad altáre Dei.}{%
℣. Je monterai à l'autel de Dieu.}

\versio{%
℟. Ad Deum, qui lætíficat iuventútem meam.}{%
℟. Jusqu'au Dieu qui réjouit ma jeunesse.}

\versio{%
℣. Iúdica me, Deus, et discérne causam meam de gente non sancta : ab hómine iníquo et dolóso érue me.}{%
℣. Jugez-moi, ô Dieu, et distinguez ma cause de celles de la nation impie : arrachez-moi de l'homme inique et trompeur.}

\versio{%
℟. Quia tu es, Deus, fortitúdo mea : quare me repulísti, et quare tristis incédo, dum afflígit me inimícus ?}{%
℟. Car, ô Dieu, vous êtes ma force : pourquoi m'avez-vous repoussé, et m'en vais-je triste, tandis que l'ennemi m'af\-flige ?\looseness=-1}

\versio{%
℣. Emítte lucem tuam et veritátem tuam : ipsa me deduxérunt et adduxérunt in montem sanctum tuum, et in tabernácula tua.\looseness=-1}{%
℣. Envoyez votre lumière et votre vérité : elles m'ont conduit et m'ont amené à votre montagne sainte et dans vos palais.}

\versio{%
℟. Et introíbo ad altáre Dei : ad Deum qui lætíficat iuventútem meam.}{%
℟. Et je monterai à l'autel de Dieu, jusqu'au Dieu qui réjouit ma jeunesse.}

\versio{%
℣. Confitébor tibi in cíthara, Deus, Deus meus : quare tristis es anima mea, et quare contúrbas me ?}{%
℣. Je vous louerai sur la harpe, ô Dieu, mon Dieu : pourquoi es-tu triste, ô mon âme, et pourquoi me troubles-tu ?\looseness=-1}

\versio{%
℟. Spera in Deo, quóniam adhuc confitébor illi : salutáre vultus mei, et Deus meus.}{%
℟. Espère en Dieu, parce que je le louerai encore : il est le salut de mon visage et il est mon Dieu.}

\versio{%
℣. Glória Patri, et Fílio, et Spirítui Sancto.}{%
℣. Gloire au Père, et au Fils, et au Saint-Esprit.}%

\versio
{
℟. Comme il était au commencement, maintenant, et toujours, et dans les siècles des siècles. Ainsi \mbox{soit-il.}}

\versio{%
℣. Introíbo ad altáre Dei.}{%
℣. Je monterai à l'autel de Dieu.}

\versio{%
℟. Ad Deum qui lætíficat iuventútem meam.}{%
℟. Jusqu'au Dieu qui réjouit ma jeunesse.}

\versio{%
℣. Adiutórium nostrum in nómine Dómini.}{%
℣. Notre secours est dans le nom du Seigneur.}

\versio{%
℟. Qui fecit cælum et terram.}{%
℟. Qui a fait le ciel et la terre.}

\rubrica{%
Le célébrant récite le \emph{Confiteor}, puis les ministres (et les fidèles à la messe lue) répondent :%
}

\versio{%
℟. Misereátur tui omnípotens Deus, et dimissis peccátis tuis, perdúcat te ad vitam ætérnam.}{%
℟. Que le Dieu tout-puissant vous fasse miséricorde, vous pardonne vos péchés, et vous con\-duise à la vie \mbox{éternelle.}}

\versio{%
℣. Amen.}{%
℣. Ainsi soit-il.}

\rubrica{%
Les ministres récitent alors le \emph{Confiteor} :%
}

\versio{%
Confíteor Deo omnipoténti, beá\-tæ Maríæ semper Vírgini, beáto Michaéli Archángelo, beáto Ioánni Baptístæ, sanctis Apóstolis Petro et Páulo, ómnibus Sanctis, et tibi pater : quia peccávi nimis cogitatióne, verbo, et ópere : mea culpa, mea culpa, mea máxima culpa. Ideo precor beátam Maríam semper Vírginem, beátum Michaélem Archángelum, beátum Ioánnem Baptístam, sanctos Apóstolos Petrum et Páulum, omnes Sanctos, et te, pater, oráre pro me ad Dóminum Deum nostrum.}{%
Je confesse à Dieu tout-puissant, à la bienheureuse Marie toujours vierge, à saint Michel Archange, à saint Jean-Baptiste, aux saints Apôtres Pierre et Paul, à tous les saints et à vous mon père, que j'ai beaucoup péché, par pensées, par paroles et par actions. C'est ma faute, c'est ma faute, c'est ma très grande faute. C'est pourquoi je supplie la bienheureuse Marie toujours vierge, saint Michel Archange, saint Jean-Baptiste, les saints Apôtres Pierre et Paul, tous les saints et vous mon père, de prier pour moi le Seigneur notre Dieu.}


\versio{%
℣. Misereátur vestri omnípotens Deus, et dimíssis peccátis vestris, perdúcat vos ad vitam ætérnam.}{%
℣. Que le Dieu tout-puissant vous fasse miséricorde, qu'il vous pardonne vos péchés et vous con\-duise à la vie éternelle.}

\versio{%
℟. Amen.}{%
℟. Ainsi soit-il.}

\versio{%
℣. Indulgéntiam, absolutiónem, et remissiónem peccatórum nostrórum, tríbuat nobis omnípotens et miséricors Dóminus.}{%
℣. Que le Dieu tout-puissant et miséricordieux nous accorde le pardon, l'absolution et la rémission de nos péchés.}

\versio{%
℟. Amen.}{%
℟. Ainsi soit-il.}

\versio{%
℣. Deus, tu convérsus vivificábis nos.}{%
℣. Dieu, tournez-vous vers nous et donnez-nous la vie.}

\versio{%
℟. Et plebs tua lætábitur in te.}{%
℟. Votre peuple se réjouira en vous.\looseness=-1}

\versio{%
℣. Osténde nobis Dómine, misericórdiam tuam.}{%
℣. Montrez-nous, Seigneur, votre miséricorde.}

\versio{%
℟. Et salutáre tuum da nobis.}{%
℟. Accordez-nous votre salut.}

\versio{%
℣. Dómine, exáudi oratiónem meam.\looseness=-1}{%
℣. Seigneur, exaucez ma prière.}

\versio{%
℟. Et clamor meus ad te véniat.}{%
℟. Que mon appel vous parvienne.\looseness=-1}

\versio{%
℣. Dóminus vobíscum.}{%
℣. Le Seigneur soit avec vous.}

\versio{%
℟. Et cum spíritu tuo.}{%
℟. Et avec votre esprit.}

\medskip

\versio{%
Orémus.}{%
Prions.}

\subsection*{Montée à l'autel}

\rubrica{En montant à l'autel le prêtre dit la prière suivante :}

\versio{%
Aufer a nobis, quǽsumus, Dómine, iniquitátes nostras : ut ad Sancta sanctórum puris mereámur méntibus introíre. Per Christum Dóminum no\-strum. Amen.}{%
Enlevez nos fautes, Seigneur, pour que nous puissions pénétrer dans le Saint des Saints avec une âme pure. Par le Christ notre Seigneur. Ainsi soit-il.}

%\smallskip\needspace{\baselineskip}
\rubrica{L'autel, en pierre, représente le Christ : \emph{il est la pierre vivante choisie par Dieu} (I~Pierre, 2,4). Cette pierre contient les reliques des martyrs qui ont versé leur sang pour le Christ. En la baisant le prêtre adore le Christ et vénère les saints martyrs.}

\versio{%
Orámus te, Dómine, per mérita Sanctórum tuórum, quorum relíquiæ hic sunt, et ómnium Sanctórum : ut indúlgere dignéris ómnia peccáta mea. Amen.}{%
Nous vous en prions, Seigneur, par les mérites de vos saints (il baise l'autel au milieu) dont nous avons ici les reliques et de tous les saints, daignez pardonner tous mes péchés. Ainsi soit-il.}

\needspace{3\baselineskip}
\rubrica{Le Prêtre bénit l'encens, en disant :}

\versio{%
Ab illo bene ✠ dicáris, in cuius honóre cremáberis. Amen.}{%
Sois bé ✠ ni par celui en l'honneur de qui tu vas brûler. Ainsi soit-il.}

\rubrica{Et ayant reçu l'encensoir du diacre, il encense l'autel, sans rien dire. Puis le diacre, ayant reçu l'encensoir du Prêtre, encense ce dernier.}


\subsection*{Avant-messe}

\rubrica{La première partie de la Liturgie est appelée "Avant-messe" ou "messe des catéchumènes". Elle est une préparation au saint Sacrifice et consiste dans un ensemble de prières, de chants et de lectures de la Sainte Écriture. Chaque jour liturgique a un thème ou une orientation spirituelle particulière qui y est exprimée. \emph{Les chants sacrés qui résument les plus saintes vérités préparent harmonieusement nos âmes aux mystères que nous devons célébrer. Ils nous mettent à l'unisson des chants divins et nous accordent aux réalités divines, avec nous-mêmes et les uns aux autres : nous ne formons plus qu'un chœur unique et homogène d'hommes saints}. (Denys l'Aréopagite)}


\subsection*{%
Introït%
}

\rubrica{Voir au propre du jour.}


\subsection*{%
Kyrie%
}

\rubrica{Le Kyrie est le reste d'une litanie plus longue, qui était toujours chantée en grec. Le diacre formulait des intentions et on répondait : Seigneur, ayez pitié. De cette litanie il ne reste aujourd'hui que neuf invocations, trois pour chaque personne de la Sainte Trinité. Nous recommandons à Dieu les besoins de l'Église et nos intentions personnelles.}

\versio{%
Kýrie, eléison. \emph{ter}

Christe, eléison. \emph{ter}

Kýrie, eléison. \emph{ter}%
}{%
Seigneur, ayez pitié. \emph{ter}

Christ, ayez pitié. \emph{ter}

Seigneur, ayez pitié. \emph{ter}%
}%


\needspace{5\baselineskip}
\subsection*{%
Gloria%
}%

\rubrica{Le Gloria faisait partie de l'office de nuit et fut progressivement introduit dans la messe à partir du V\ieme\ siècle. C'est un développement du chant des anges dans la nuit de Noël. C'est la grande glorification du Dieu trinitaire : Gloire au Père, au Fils et au Saint-Esprit.}

\versio{%
Glória in excélsis Deo. Et in terra pax homínibus bonæ voluntátis. Laudámus te. Benedícimus te. Adorámus te. Glorificámus te. Grátias ágimus tibi propter magnam glóriam tuam. Dómine Deus, Rex cæléstis, Deus Pater omnípotens. Dómine Fili unigénite, Iesu Christe. Dómine Deus, Agnus Dei, Fílius Patris. Qui tollis peccáta mundi, miserére nobis. Qui tollis peccáta mundi, súscipe deprecatiónem nostram. Qui sedes ad déxteram Patris, miserére nobis. Quóniam tu solus Sanctus. Tu solus Dóminus. Tu solus Altíssimus, Iesu Christe. Cum Sancto Spíritu in glória Dei Patris. Amen.%
}{%
Gloire à Dieu au plus haut des cieux, et paix sur la terre aux hommes de bonne volonté. Nous vous louons, nous vous bénissons, nous vous adorons, nous vous glorifions. Nous vous rendons grâces pour votre gloire immense, Seigneur Dieu, Roi du ciel, Dieu Père tout-puissant. Seigneur Fils unique, Jésus-Christ, Seigneur Dieu, Agneau de Dieu, Fils du Père, vous qui enlevez les péchés du monde, ayez pitié de nous ; vous qui enlevez les péchés du monde, accueillez notre prière ; vous qui siégez à la droite du Père, ayez pitié de nous. Car vous êtes le seul Saint, le seul Seigneur, le seul Très-Haut, Jésus-Christ, avec le Saint-Esprit, dans la gloire de Dieu le Père. Ainsi soit-il.\looseness=-1%
}%


\subsection*{Collecte}

\rubrica{Le prêtre se tourne vers les fidèles pour les saluer et les inviter à s'unir à la prière qu'il va prononcer en leur nom.}

\versio{%
℣. Dóminus vobíscum.}{%
℣. Le Seigneur soit avec vous.}

\versio{%
℟. Et cum spíritu tuo.}{%
℟. Et avec votre esprit.}

\rubrica{La Collecte chantée par le prêtre exprime l'intention de la prière de l'Église en ce jour et la grâce spéciale qu'elle demande. Cette oraison est répétée plusieurs fois au cours de l'office. Le prêtre la prononce les mains élevées comme pour diriger la prière vers le ciel ; dans la Sainte Écriture élever les mains est synonyme de prier. On se tient debout, sauf les jours de pénitence et aux messes des défunts, où l'on est à genoux.}

%\medskip\pagebreak[1]

\versio{%
Orémus.}{%
Prions.}

\rubrica{Voir au propre du jour. À la fin de la collecte, on répond :}

\versio{%
℟. Amen.}{%
℟. Ainsi soit-il.}


\needspace{5\baselineskip}
\subsection*{Épître}

\rubrica{Les lectures ne suivent pas l'ordre des livres de la Bible, mais l'Église a choisi les textes les plus importants en rapport avec le temps liturgique ou le mystère célébré en ce jour.}

\rubrica{Voir à la Messe du jour.}

\rubrica{À la messe lue on répond :}

\versio{%
℟. Deo grátias.}{%
℟. Rendons grâces à Dieu.}%


\subsection*{Graduel, Alléluia et Séquence}

\rubrica{Graduel, Alléluia sont des chants de méditation sur les lectures ou le mystère liturgique de ce jour. \emph{Alleluia} signifie "louez Dieu". L'Alléluia se prolonge dans une mélodie ornée qui exprime la joie spirituelle au-delà des paroles.}

\rubrica{Voir au propre du jour.}


\subsection*{%
Évangile%
}

\rubrica{On chante le saint Évangile tourné vers le Nord (le chœur étant normalement orienté vers l'Est) parce que le Nord symbolise les âmes qui se trouvent dans la froideur de l'infidélité et de l'ignorance du Christ, et qui sont éveillées par le souffle du Saint-Esprit.}

\versio{%
℣. Dóminus vobíscum.}{%
℣. Le Seigneur soit avec vous.}

\versio{%
℟. Et cum spíritu tuo.}{%
℟. Et avec votre esprit.}

\needspace{3\baselineskip}
\rubrica{Voir au propre du jour. À la messe lue on répond :}

\versio{%
℟. Laus tibi, Christe.}{%
℟. Louange à vous, ô Christ.}


\subsection*{%
Credo%
}

\versio{%
Credo in Deum, Patrem omnipoténtem, factórem cæli et terræ, visibilium omnium et invisibilium.

Et in unum Dóminum Iesum Christum, Fílium Dei unigenitum. Et ex Patre natum ante ómnia sǽcula. Deum de Deo, lumen de lúmine, Deum verum de Deo vero. Génitum, non factum, consubstantiálem Patri : per quem ómnia facta sunt. Qui propter nos hómines et propter nostram salútem descéndit de cælis. Et incarnátus est de Spíritu Sancto ex María Vírgine : et homo factus est. Crucifíxus étiam pro nobis : sub Póntio Piláto passus, et sepúltus est. Et resurréxit tértia die, secúndum Scriptúras. Et ascéndit in cælum : sedet ad déxteram Patris. Et iterum ventúrus est cum glória iudicáre vivos et mórtuos : cuius regni non erit finis.\looseness=-1

Et in Spíritum Sanctum, Dóminum, et vivificántem : qui ex Patre Filióque procédit. Qui cum Patre et Fílio simul adorátur et conglorificátur : qui locútus est per Prophétas.

Et unam, sanctam, cathólicam et apostólicam Ecclésiam. Confíteor unum baptísma in remissiónem peccatórum. Et exspécto resurrectiónem mortuórum. Et vitam ventúri sǽculi. Amen.%
}{%
  Je crois en un seul Dieu, le Père tout-puissant, créateur du ciel et de la terre, de toutes choses, visibles et invisibles.

  Je crois en un seul Seigneur Jésus-Christ, le Fils unique de Dieu, né du Père avant tous les siècles ; Dieu né de Dieu, lumière née de la lumière, vrai Dieu né du vrai Dieu. Engendré, non pas créé, consubstantiel au Père, et par qui tout a été créé. C'est lui qui, pour nous les hommes, et pour notre salut, est descendu des cieux. Il a pris chair de la Vierge Marie par l'action du Saint-Esprit, et il s'est fait homme. Puis il fut crucifié pour nous sous Ponce-Pilate : il souffrit et fut mis au tombeau. Il ressuscita le troisième jour, suivant les Ecritures. Il monta aux cieux, où il siège à la droite du Père. De nouveau il viendra dans la gloire pour juger les vivants et les morts, et son règne n'aura pas de fin.

  Je crois en l'Esprit-Saint, qui est Seigneur et qui donne la vie, qui procède du Père et du Fils. Avec le Père et le Fils, il reçoit même adoration et même gloire. Il a parlé par les prophètes.

  Je crois à l'Église une, sainte, catholique et apostolique. Je reconnais un seul Baptême pour la rémission des péchés, et j'attends la résurrection des morts et la vie du monde à venir. Ainsi soit-il.
}


\needspace{5\baselineskip}
\subsection*{%
Offertoire%
}\blindsection{Offertoire}

\rubrica{À l'Offertoire l'Église offre le Corps et le Sang du Christ, représentés par le pain et le vin. Notre Seigneur a communiqué à son Église le pouvoir d'offrir le même sacrifice qu'il offrit sur la Croix. L'Église s'y unit comme victime. \emph{Le Christ lui-même est à la fois le prêtre qui offre et la victime qui est offerte. Le sacrifice de l'Église est le sacrement quotidien du sacrifice du Christ. L'Église, qui est le corps dont il est la tête, s'y offre elle-même par lui. C'est pourquoi dans ce sacrement l'Église elle-même est offerte dans ce qu'elle offre.} (S.~Augustin). Nous sommes donc une seule victime avec le Christ, unissant nos propres offrandes et nos sacrifices à celui du Christ et de l'Église. Notre participation à cette offrande est exprimée par le chant de l'offertoire.}

\versio{%
℣. Dóminus vobíscum.}{%
℣. Le Seigneur soit avec vous.}

\versio{%
℟. Et cum spíritu tuo.}{%
℟. Et avec votre esprit.}

\medskip

\versio{%
Orémus.}{%
Prions.}


\vspace{-.6\baselineskip}
\subsection*{Antienne d'offertoire}


\rubrica{Voir au propre du jour.}


\vspace{-.6\baselineskip}
\subsection*{Offrande du pain}

\versio{%
Súscipe, sancte Pater, omnípotens ætérne Deus, hanc immaculátam hóstiam, quam ego indígnus fámulus tuus óffero tibi Deo meo, vivo et vero, pro innumerabílibus peccátis, et offensiónibus, et negligéntiis meis, et pro ómnibus circumstántibus, sed et pro ómnibus fidélibus christiánis vivis atque defúnctis : ut mihi et illis profíciat ad salútem in vitam ætérnam. Amen.}{%
Recevez, Père saint, Dieu éternel et tout-puissant, cette offrande sans tache, que moi, votre indigne serviteur, je vous présente, à vous, mon Dieu vivant et vrai, pour mes péchés, offenses et négligences sans nombre, pour tous ceux qui m'entourent, ainsi que pour tous les fidèles vivants et morts : qu'elle serve à mon salut et au leur pour la vie éternelle. Ainsi soit-il.}

\vspace{-.8\baselineskip}
\subsection*{Préparation et offrande du vin}

\rubrica{Le diacre prépare le calice en y versant le vin ; le sous-diacre y ajoute quelques gouttes d'eau que le prêtre bénit (s'il est seul le prêtre accomplit lui-même ces cérémonies). L'eau mêlée au vin figure l'union en Jésus-Christ de la nature humaine à la personne divine, et symbolise aussi le chrétien intimement uni au Christ. Le vin et l'eau rappellent le Sang et l'eau qui coulèrent du côté du Christ percé par la lance. \emph{Quand le vin et l'eau sont mêlés dans le calice, le peuple est uni au Christ et la foule des croyants est associée et réunie à celui en qui elle croit.} (S.~Cyprien)}

\versio{%\widowpenalty=10000%
Deus, ✠ qui humánæ substántiæ dignitátem mirabíliter condidísti, et mirabílius reformásti : da nobis, per huius aquæ et vini mystérium, eius divinitátis esse consórtes, qui humanitátis nostræ fíeri dignátus est párticeps, Iesus Christus, Fílius tuus, Dóminus noster : Qui tecum vivit et regnat in unitáte Spíritus Sancti Deus : per ómnia sǽcula sæculórum. Amen.\looseness=1%
}{%\widowpenalty=10000%
Dieu, ✠ qui, d'une manière admirable, avez créé la nature humaine dans sa noblesse, et l'avez restaurée d'une manière plus admirable encore, accordez-nous, selon le mystère de cette eau et de ce vin, de prendre part à la divinité de celui qui a daigné partager notre humanité, Jésus-Christ votre Fils, Notre Seigneur, qui, étant Dieu, vit et règne avec vous en l'unité du Saint-Esprit, dans tous les siècles des siècles. \mbox{Ainsi soit-il}.\looseness=1}

\versio{%
Offérimus tibi, Dómine, cálicem salutáris, tuam deprecántes cleméntiam : ut in conspéctu divínæ maiestátis tuæ, pro nostra et totíus mundi salúte, cum odóre suavitátis ascéndat. Amen.}{%
Nous vous offrons, Seigneur, le calice du salut, et nous demandons à votre clémence qu'il s'élève en parfum agréable devant votre divine Majesté, pour notre salut et celui du monde entier. Ainsi soit-il.}

\versio{%
In spíritu humilitátis, et in ánimo contríto suscipiámur a te, Dómine : et sic fiat sacrifícium nostrum in conspéctu tuo hódie, ut pláceat tibi, \mbox{Dómine Deus.}}{%
Voyez l'humilité de nos âmes et la contrition de nos cœurs : accueillez-nous, Seigneur, et que notre sacrifice s'accomplisse aujourd'hui devant vous de telle manière qu'il vous soit agréable, Seigneur Dieu.}

\versio{%
Veni, sanctificátor, omnípotens ætérne Deus : et béne ✠ dic hoc sacrifícium, tuo sancto nómini præparátum.}{%
Venez, Sanctificateur, Dieu éternel et tout-puissant, et ✠ bénissez ce sacrifice préparé pour votre saint Nom.}

\subsection*{Encensement}

\rubrica{Comme l'encens, consumé par le feu, s'élève en vapeur odorante, ainsi l'Église avec le Christ, consumée par le feu du Saint-Esprit, s'élève en sacrifice envers Dieu. On encense les oblats (le pain et le vin), puis l'autel, le prêtre, et enfin le clergé et les fidèles.}

\versio{%
Per intercessiónem beáti Michaélis archángeli, stantis a dextris altáris incénsi, et ómnium electórum suórum, incénsum istud dignétur Dóminus bene ✠ dícere, et in odórem suavitátis accípere. Per Christum Dóminum nostrum. Amen.}{%
Par l'intercession de l'archange saint Michel, qui se tient à droite de l'autel de l'encens, et par l'intercession de tous ses élus, que le Seigneur daigne bénir ✠ cet encens, et le recevoir comme un parfum agréable. Par le Christ notre Seigneur. Ainsi soit-il.}

\versio{%
Incénsum istud, a te benedíctum, ascéndat ad te, Dómine : et descéndat super nos misericórdia tua. }{%
Que cet encens béni par vous, Seigneur, monte vers vous, et que descende sur nous votre miséricorde.}

\versio{%
Dirigátur, Dómine, orátio mea, sicut incénsum, in conspéctu tuo : elevátio mánuum meárum sacrifícium vespertínum. Pone, Dómine, custódiam ori meo, et óstium circumstántiæ lábiis meis : ut non declínet cor meum in verba malítiæ, ad excusándas excusatiónes in peccátis.}{%
Seigneur, que ma prière s'élève comme l'encens devant votre face ; que mes mains levées soient comme l'offrande du soir. Placez, Seigneur, une garde à ma bouche et une barrière tout autour de mes lèvres. Que mon cœur ne se porte pas à des paroles mauvaises qui servent de \mbox{prétexte au péché. (Ps. 140)}}

\versio{%
Accéndat in nobis Dóminus ignem sui amóris, et flammam ætérnæ caritátis. Amen.}{%
Que le Seigneur allume en nous le feu de son amour et la flamme de l'éternelle charité. Ainsi soit-il.}


\subsection*{Lavabo}

\rubrica{\emph{Les mains sont le symbole de l'action. En les lavant nous signifions la pureté et l'innocence des actions.} (S.~Cyprien)}

\versio{%
Lavábo inter innocéntes manus meas : et circúmdabo altáre tuum, Dómine : Ut áudiam vocem laudis, et enárrem univérsa mirabília tua. Dómine, diléxi decórem domus tuæ, et locum habitatiónis glóriæ tuæ. Ne perdas cum ímpiis, Deus, ánimam meam, et cum viris sánguinum vitam meam : In quorum mánibus iniquitátes sunt : déxtera eórum repléta est munéribus. Ego autem in innocéntia mea ingréssus sum : rédime me, et miserére mei. Pes meus stetit in dirécto : in ecclésiis benedícam te, Dómine. Glória Patri, et Fílio, et Spirítui Sancto, Sicut erat in princípio, et nunc, et semper, et in sǽcula sæculórum. Amen.}{%
Je laverai mes mains parmi les innocents, et je me tiendrai devant votre autel, Seigneur. Pour entendre la voix de la louange et raconter toutes vos merveilles. Seigneur, j'aime la beauté de votre maison et le lieu où réside votre gloire. Ô Dieu, ne condamnez pas mon âme avec celle des impies ; ne m'enlevez pas la vie comme aux hommes de sang. Leurs mains commettent l'iniquité, et leur droite est comblée de présents.
Pour moi je marche dans l'innocence ; rachetez-moi et ayez pitié de moi. Mon pied se tient dans la voie droite ; je vous bénirai, Seigneur, dans l'assemblée. Gloire au Père, au Fils, et au Saint-Esprit. Comme il était au commencement, maintenant et toujours, et dans tous les siècles des siècles. Ainsi soit-il.}


%\vfill\pagebreak[0]

\subsection*{Prière à la Sainte Trinité}

\versio{%
Súscipe, sancta Trínitas, hanc oblatiónem, quam tibi offérimus ob memóriam passiónis, resurrectiónis, et ascensiónis Iesu Christi Dómini nostri : et in honórem beátæ Maríæ semper Vírginis, et beáti Ioánnis Baptístæ, et sanctórum Apostolórum Petri et Pauli, et istórum, et ómnium sanctórum : ut illis profíciat ad honórem, nobis autem ad salútem : et illi pro nobis intercédere dignéntur in cælis, quorum memóriam ágimus in terris. Per eúndem Christum Dóminum nostrum. Amen.}{%
Recevez, Trinité Sainte, cette of\-frande que nous vous présentons en mémoire de la Passion, de la Résurrection et de l'Ascension de Jésus-Christ notre Seigneur, en l'honneur aussi de la bienheureuse Marie toujours vierge, de saint Jean-Baptiste, des saints Apôtres Pierre et Paul, des saints dont les reliques sont ici, et de tous les saints. Qu'elle soit pour eux une source d'honneur, et pour nous une cause de salut, et qu'ils daignent intercéder pour nous au ciel, eux dont nous célébrons la mémoire sur terre. Par le Christ notre Seigneur. Ainsi soit-il.\looseness=-1}

\rubrica{Le prêtre dit, tourné vers les fidèles :}

\versio{%
Oráte, fratres : ut meum ac vestrum sacrifícium acceptábile fiat apud Deum Patrem omnipoténtem.}{%
Priez, mes frères, pour que mon sacrifice, qui est aussi le vôtre, puisse être agréé par Dieu le Père tout-puissant.}

\rubrica{À la messe lue on répond :}

\versio{%
℟. Suscípiat Dóminus sacrifícium de mánibus tuis, ad laudem et glóriam nóminis sui, ad utilitátem quoque nostram, totiúsque Ecclésiæ suæ sanctæ.}{%
℟. Que le Seigneur reçoive de vos mains le sacrifice, à la louange et à la gloire de son Nom, ainsi que pour notre bien et celui de toute sa sainte Église.}


\subsection*{%
Secrète%
}

\rubrica{Voir au propre du jour.}


\subsection*{Oblation du Sacrifice}

\rubrica{Commence ici la partie la plus sacrée de la Liturgie : le sacrifice réel et sacramentel où le prêtre immole le Christ. \emph{Vraiment à cette heure très redoutable il faut tenir haut son cœur vers Dieu, et non en bas vers la terre et les affaires terrestres. D'autorité donc le pontife  enjoint à cette heure à tous de laisser de côté les soucis, les sollicitudes domestiques, et de tenir leur cœur au ciel vers Dieu ami des hommes.} \mbox{(S. Cyrille de Jérusalem)}}


\subsection*{%
Préface%
}


\rubrica{Par le chant de la Préface le prêtre rend grâces à Dieu au nom de l'Église pour l'œuvre du salut réalisée par Jésus-Christ.}

\versio{%
℣. Per ómnia sǽcula sæculórum. ℟. Amen.}{%
℣. Dans tous les siècles des siècles. ℟. Ainsi soit-il.}

\versio{%
℣. Dóminus vobíscum.}{%
℣. Le Seigneur soit avec vous.}

\versio{%
℟. Et cum spíritu tuo.}{%
℟. Et avec votre esprit.}

\versio{%
℣. Sursum corda.}{%
℣. Élevons nos cœurs.}

\versio{%
℟. Habémus ad Dóminum.}{%
℟. Ils sont tournés vers le Seigneur.\looseness=-1}

\versio{%
℣. Grátias agámus Dómino Deo nostro.}{%
℣. Rendons grâces au Seigneur notre Dieu.}

\versio{%
℟. Dignum et iustum est.}{%
℟. C'est juste et nécessaire.}

\rubrica{Voir au propre du jour.}


\needspace{9\baselineskip}
\subsection*{%
Sanctus%
}

\rubrica{L'hymne au Dieu trois fois saint se compose d'une vision du prophète Isaïe (Is. 6,3) et des acclamations chantées lors de l'entrée triomphale du Christ à Jérusalem. Le sacrifice du Christ s'achève par sa résurrection et son ascension : Notre-Seigneur entre triomphalement dans la Jérusalem céleste. \emph{Le Christ est entré une fois pour toutes dans le sanctuaire du ciel avec son propre Sang, nous ayant acquis une rédemption éternelle.} (Hébreux 9,12). \emph{Nous chantons cette glorification qui nous a été transmise des Séraphins, pour que, en communiant  à cette hymne, nous soyons unis aux armées angéliques.} (S.~Cyrille de Jérusalem)}

\versio{%
Sanctus, sanctus, sanctus Dóminus, Deus Sábaoth. Pleni sunt cæli et terra glória tua. Hosánna in excélsis ! Benedíctus qui venit in nómine Dómini. Hosánna in excélsis !%
}{%
Saint, saint, saint le Seigneur, Dieu des armées. Les cieux et la terre sont remplis de votre gloire ; hosanna au plus haut des cieux ! Béni soit celui qui vient au nom du Seigneur ; hosanna au plus haut des cieux !%
}%


\needspace{4\baselineskip}
\subsection*{Canon}\blindsection{Canon}

\rubrica{Canon signifie règle. Cette partie de la messe est une règle intangible. Les prières du Canon remontent au-delà du VI\ieme{} siècle, voire pour certaines à l'Apôtre Pierre lui-même.}

\rubrica{Jusqu'à la fin du Canon le silence le plus complet règne dans l'église. Le prêtre "entre" dans le saint des saints de la Liturgie. Dans les premiers siècles un rideau dérobait le prêtre à la vue de l'assemblée. Depuis le XI\ieme{} siècle le Canon est dit à voix basse par respect et vénération et aussi parce que le prêtre seul, de par son caractère sacerdotal, a le pouvoir de le prononcer et d'accomplir le sacrifice par la consécration du pain et du vin au Corps et au Sang du Christ.}

\versio{%
Te ígitur, clementíssime Pater, per Iesum Christum Fílium tuum Dóminum nostrum, súpplices rogámus, ac pétimus, uti accépta hábeas, et benedícas hæc ✠ dona, hæc ✠ múnera, hæc sancta ✠ sacrifícia illibáta. In primis, quæ tibi offérimus pro Ecclésia tua sancta cathólica : quam pacificáre, custodíre, adunáre et régere dignéris toto orbe terrárum : una cum fámulo tuo Papa nostro N. et Antístite nostro N. et ómnibus orthodóxis, atque cathólicæ et apostólicæ fídei cultóribus.}{%
Père très clément, nous vous prions humblement et nous vous demandons par Jésus-Christ votre Fils, notre Seigneur, d'accepter et de bénir ces ✠ dons, ces ✠ présents, ces ✠ offrandes saintes et sans tache. Tout d'abord, nous vous les offrons pour votre sainte Église catholique — daignez, à travers le monde entier, lui donner la paix, la protéger, la rassembler dans l'unité et la gouverner —, et aussi pour votre serviteur notre pape N., pour notre évêque N., et pour tous ceux qui, fidèles à la vraie doctrine, ont la garde de la foi catholique et apostolique.}

%\versio{%
%In primis, quæ tibi offérimus pro Ecclésia tua sancta cathólica : quam pacificáre, custodíre, adunáre et régere dignéris toto orbe terrárum : una cum fámulo tuo Papa nostro N. et Antístite nostro N. et ómnibus orthodóxis, atque cathólicæ et apostólicæ fídei cultóribus.}{%
%Tout d'abord, nous vous les offrons pour votre sainte Église catholique — daignez, à travers le monde entier, lui donner la paix, la protéger, la rassembler dans l'unité et la gouverner —, et aussi pour votre serviteur notre pape N., pour notre Evêque N., et pour tous ceux qui, fidèles à la vraie doctrine, ont la garde de la foi catholique et apostolique.}

\versio{%
Meménto, Dómine, famulórum famularúmque tuárum N. et N., et ómnium circumstántium, quorum tibi fides cógnita est et nota devótio, pro quibus tibi offérimus : vel qui tibi ófferunt hoc sacrifícium laudis, pro se suísque ómnibus : pro redemptióne animárum suárum, pro spe salútis et incolumitátis suæ : tibíque reddunt vota sua ætérno Deo, vivo et vero.}{%
Souvenez-vous, Seigneur, de vos serviteurs et de vos servantes N. et N., et de tous ceux qui nous entourent : vous connaissez leur foi, vous avez éprouvé leur attachement. Nous vous offrons pour eux, ou ils vous offrent eux-mêmes, ce sacrifice de louange pour eux et pour tous les leurs : afin d'obtenir la rédemption de leur âme, la sécurité et le salut dont ils ont l'espérance ; et ils vous adressent leurs prières, à vous, Dieu éternel, vivant et vrai.}

\versio{%
Communicántes, et memóriam venerántes, in primis gloriósæ semper Vírginis Maríæ, Genetrícis Dei et Dómini nostri Iesu Christi : sed et beáti Ioseph, eiúsdem Vírginis sponsi, et beatórum Apostolórum ac Mártyrum tuórum, Petri et Pauli, Andréæ, Iacóbi, Ioánnis, Thomæ, Iacóbi, Philíppi, Bartholomǽi, Matthǽi, Simónis et Thaddǽi : Lini, Cleti, Cleméntis, Xysti, Cornélii, Cypriáni, Lauréntii, Chrysógoni, Ioánnis et Pauli, Cosmæ et Damiáni : et ómnium Sanctórum tuórum; quorum méritis precibúsque concédas, ut in ómnibus protectiónis tuæ muniámur auxílio. Per eúndem Christum Dóminum nostrum. Amen.}{%
Unis dans une même communion, nous vénérons d'abord la mémoire de la glorieuse Marie toujours vierge, mère de notre Dieu et Seigneur Jésus-Christ, puis celle du bienheureux Joseph, l'Époux de la Vierge, de vos bienheureux Apôtres et Martyrs, Pierre et Paul, André, Jacques, Jean, Thomas, Jacques, Philippe, Barthélémy, Matthieu, Simon et Jude, Lin, Clet, Clément, Xyste, Corneille, Cyprien, Laurent, Chrysogone, Jean et Paul, Côme et Damien, et de tous vos saints. Par leurs mérites et leurs prières, accordez-nous en toute occasion le secours de votre force et de votre protection. Par le Christ notre Seigneur. Ainsi soit-il.}

%\vspace{\baselineskip}
\rubrica{À cet instant commence le rite consécratoire proprement dit. Il convient de se tenir à genoux. Le prêtre étend les mains sur le pain et le vin, comme faisait le Grand Prêtre de l'Ancien Testament sur la victime que l'on immolait pour l'expiation des péchés. Jésus-Christ a pris sur lui nos péchés pour les expier.}

\versio{%
Hanc ígitur oblatiónem servitútis nostræ, sed et cunctæ famíliæ tuæ, quǽsumus, Dómine, ut placátus accípias : diésque nostros in tua pace dispónas, atque ab ætérna damnatióne nos éripi, et in electórum tuórum iúbeas grege numerári. Per Christum Dóminum nostrum. Amen.}{%
Voici donc l'offrande que nous vous présentons, nous vos serviteurs, et avec nous votre famille entière : acceptez-la, Seigneur, avec bienveillance ; disposez dans votre paix les jours de notre vie ; veuillez nous arracher à l'éternelle damnation et nous compter au nombre de vos élus. Par le Christ notre Seigneur. Ainsi \mbox{soit-il}.}

\versio{%
Quam oblatiónem tu, Deus, in ómnibus, quǽsumus, bene ✠ díctam, adscríp ✠ tam, ra ✠ tam, rationábilem, acceptabilémque fácere dignéris : ut nobis Cor ✠ pus, et San ✠ guis fiat dilectíssimi Fílii tui Dómini nostri Iesu Christi.}{%
Cette offrande, daignez, vous, notre Dieu, la ✠ bénir, l'~✠~agréer, l'~✠~approuver pleinement, la rendre parfaite et digne de vous plaire ; qu'elle devienne ainsi pour nous le ✠ Corps et le ✠ Sang de votre Fils bien-aimé, notre Seigneur Jésus-Christ.}

\rubrica{Le prêtre récite alors l'histoire de l'institution de l'eucharistie en accomplissant les mêmes gestes que le Christ.}

\versio{%
Qui prídie quam paterétur, accépit panem in sanctas ac venerábiles manus suas, et elevátis óculis in cælum ad te Deum Patrem suum omnipoténtem, tibi grátias agens, bene ✠ díxit, fregit, dedítque discípulis suis, dicens : Accípite, et manducáte ex hoc omnes.}{%
Celui-ci, la veille de sa Passion, prit du pain dans ses mains saintes et adorables, et, les yeux levés au ciel vers vous, Dieu, son Père tout-puissant, vous rendant grâces, il ✠ bénit ce pain, le rompit et le donna à ses disciples en disant : Prenez et mangez-en tous.}


\rubrica{Le prêtre prononce alors les paroles mêmes de Notre-Seigneur. Par ces paroles il opère la conversion du pain au saint Corps du Christ.}

\versio{%
\scalebox{.91}[1]{\textsc{\addfontfeatures{Renderer=Basic}Hoc est enim Corpus meum.}}\par%
}{%
\textsc{\addfontfeatures{Renderer=Basic}Car ceci est mon Corps.}\par%
}

%\vspace{1\baselineskip plus 2\baselineskip}

\rubrica{Ces paroles étant prononcées, le prêtre fait la génuflexion pour adorer le saint Corps, l'élève pour le présenter à l'adoration des fidèles, le dépose sur l'autel et fait de nouveau la génuflexion. Le prêtre reprend le récit de l'institution de l'eucharistie :}

\versio{%
Símili modo postquam cenátum est, accípiens et hunc præclárum Cálicem in sanctas ac venerábiles manus suas : item tibi grátias agens, bene ✠ díxit, dedítque discípulis suis, dicens : Accípite, et bíbite ex eo omnes.}{%
De même, après le repas, il prit aussi ce précieux calice dans ses mains saintes et adorables, vous rendit grâces encore, le ✠ bénit et le donna à ses disciples en disant : Prenez et buvez-en tous.}

\rubrica{Prononçant alors les paroles mêmes de Notre-Seigneur, le prêtre opère la conversion du vin au précieux Sang du Christ.}
%
\nopagebreak\smallskip%
%
\versio{%
\textsc{\addfontfeatures{Renderer=Basic}Hic est enim Calix Sánguinis mei, novi et ætérni Testaménti : mystérium fídei : qui pro vobis et pro multis effundétur in remissiónem peccatórum.}%
}{%
\textsc{\addfontfeatures{Renderer=Basic}Car ceci est le Calice de mon Sang, le Sang de l'Alliance nouvelle et éternelle, le mystère de la foi, qui sera versé pour vous et pour un grand nombre en rémission des péchés.}%
}

\smallskip

\versio{%
Hæc quotiescúmque fecéritis, in mei memóriam faciétis.}{%
Toutes les fois que vous ferez cela, vous le ferez en mémoire de moi.}%

%\vspace{1\baselineskip plus 2\baselineskip}

\rubrica{Ces paroles étant prononcées, le prêtre fait la génuflexion pour adorer le précieux Sang. Il l'élève pour le présenter à l'adoration, et fait de nouveau la génuflexion.\\
\emph{Avant la consécration, ceci est du pain~; dès que sont intervenues les paroles du Christ, c'est le Corps du Christ. Avant les paroles du Christ, le calice contient du vin et de l'eau ; dès que les paroles du Christ ont agi, il contient le Sang qui a racheté le peuple.} (S. Ambroise)}

%\pagebreak[0]

\versio{%
Unde et mémores, Dómine, nos servi tui, sed et plebs tua sancta, eiúsdem Christi Fílii tui Dómini nostri tam beátæ passiónis, nec non et ab ínferis resurrectiónis, sed et in cælos gloriósæ ascensiónis : offérimus præcláræ maiestáti tuæ de tuis donis ac datis, hóstiam ✠ puram, hóstiam ✠ sanctam, hóstiam ✠ immaculátam, Panem ✠ sanctum vitæ ætérnæ, et Cálicem ✠ salútis perpétuæ.}{%
C'est pourquoi, en mémoire, Seigneur, de la bienheureuse Passion du Christ votre Fils, notre Seigneur, de sa Résurrection du séjour des morts, et aussi de sa glorieuse Ascension dans les cieux, nous, vos serviteurs, et avec nous votre peuple saint, nous présentons à votre glorieuse Majesté — offrande choisie parmi les biens que vous nous avez donnés — la victime ✠ parfaite, la victime ✠ sainte, la victime ✠ sans tache, le pain ✠ sacré de la vie éternelle et le calice ✠ de l'éternel salut.}

\versio{%
Supra quæ propítio ac seréno vultu respícere dignéris : et accépta habére, sícuti accépta habére dignátus es múnera púeri tui iusti Abel, et sacrifícium Patriárchæ nostri Abrahæ : et quod tibi óbtulit summus sacérdos tuus Melchísedech, sanctum sacrifícium, immaculátam hóstiam.}{%
Sur ces offrandes, daignez jeter un regard favorable et bienveillant ; acceptez-les, comme vous avez bien voulu accepter les présents de votre serviteur Abel le Juste, le sacrifice d'Abraham, le père de notre race, et celui que vous offrit votre souverain prêtre Melchisédech, offrande sainte, sacrifice sans tache.}

\versio{%
Súpplices te rogámus, omnípotens Deus : iube hæc perférri per manus sancti Angeli tui in sublíme altáre tuum, in conspéctu divínæ majestátis tuæ : ut, quotquot ex hac altáris participatióne sacrosánctum Fílii tui Cor ✠ pus, et Sán ✠ guinem sumpsérimus, omni benedictióne † cælésti et grátia repleámur. Per eúndem Christum Dóminum nostrum. Amen.}{%
Nous vous en supplions, Dieu tout-puissant, faites porter ces offrandes par les mains de votre saint ange, là-haut, sur votre autel, en présence de votre divine Majesté. Et quand nous recevrons, en communiant ici à l'autel, le Corps ✠ et le Sang ✠ infiniment saints de votre Fils, puissions-nous tous être comblés des grâces et des bénédictions du ciel. Par le Christ notre Seigneur. Ainsi soit-il.}

\versio{%
Meménto étiam, Dómine, famulórum famularúmque tuárum N. et N., qui nos præcessérunt cum signo fídei, et dórmiunt in somno pacis. Ipsis, Dómine, et ómnibus in Christo quiescéntibus, locum refrigérii, lucis et pacis, ut indúlgeas, deprecámur. Per eúndem Christum Dóminum nostrum. Amen.%
}{%
Souvenez-vous aussi, Seigneur, de vos serviteurs et de vos servantes N. et N., qui sont partis avant nous, marqués du sceau de la foi, et qui dorment du sommeil de la paix. À ceux-là, Seigneur, ainsi qu'à tous ceux qui reposent dans le Christ, accordez, nous vous en supplions, le séjour du bonheur, de la lumière et de la paix. Par le Christ notre Seigneur. Ainsi soit-il.}

\versio{%
Nobis quoque peccatóribus fámulis tuis, de multitúdine miseratiónum tuárum sperántibus, partem áliquam, et societátem donáre dignéris, cum tuis sanctis Apóstolis et Martýribus : cum Ioánne, Stéphano, Matthía, Bárnaba, Ignátio, Alexándro, Marcellíno, Petro, Felicitáte, Perpétua, Agatha, Lúcia, Agnéte, Cæcília, Anastásia et ómnibus Sanctis tuis : intra quorum nos consórtium, non æstimátor mériti, sed véniæ, quǽsumus, largítor admítte. Per Christum Dóminum nostrum.}{%
À nous aussi pécheurs, vos serviteurs, qui mettons notre con\-fian\-ce dans votre infinie miséricorde, daignez accorder une place dans la communauté de vos saints Apôtres et Martyrs, avec Jean, Etienne, Matthias, Barnabé, Ignace, Alexandre, Marcellin, Pierre, Félicité, Perpétue, Agathe, Lucie, Agnès, Cécile, Anastasie, et avec tous vos saints. Pour nous admettre en leur compagnie, ne pesez pas la valeur de nos actes, mais accordez-nous largement votre pardon. Par le Christ notre Seigneur. Ainsi soit-il.}


\versio{%
Per quem hæc ómnia, Dómine, semper bona creas, sanctí ✠ ficas, viví ✠ ficas, bene ✠ dícis et præstas nobis.}{%
Par lui, Seigneur, vous ne cessez de créer tous ces biens et vous les ✠ sanctifiez, vous leur donnez ✠ vie et vous les ✠ bénissez pour nous en faire don.}

\rubrica{C'est par ce Sacrifice que toute gloire est rendue à Dieu par le Christ. Le prêtre élève l'hostie et le calice, après avoir tracé avec l'hostie trois signes de croix sur le calice, puis deux signes de croix devant le calice.}

\versio{%
Per ip ✠ sum, et cum ip ✠ so, et in ip ✠ so, est tibi Deo Patri ✠ omnipoténti, in unitáte Spíritus ✠ Sancti, omnis honor, et glória.}{%
Par ✠ lui, avec ✠ lui, et en ✠ lui, vous sont donnés, Dieu Père ✠ tout-puissant, dans l'unité du ✠ Saint-Esprit, tout honneur et toute gloire.}

%\needspace{2\baselineskip}\pagebreak[1]
\rubrica{En répondant \emph{Amen} à cette conclusion du Canon, nous adhérons à l'œuvre sacrée qui vient de s'accomplir.}

\versio{%
℣. Per ómnia sǽcula sæculórum.}{%
℣. Dans tous les siècles des siècles.}

\versio{%
℟. Amen.}{%
℟. Ainsi soit-il.}


\needspace{9\baselineskip}
\subsection*{Pater  Noster}\blindsection{Communion}

\rubrica{\emph{Ô très grand amour de Dieu pour les hommes ! À ceux qui l'avaient abandonné il a accordé un tel pardon, une telle part de grâces, qu'il se fait appeler Père.} (S. Cyrille de Jérusalem). Au nom de l'Église le prêtre chante la prière que Jésus-Christ nous a apprise. Dans les rites latins le Notre Père a toujours été chanté par le prêtre seul.}

\versio{%
Orémus. Præcéptis salutáribus móniti, et divína institutióne formáti, audémus dícere :}{%
Prions. Éclairés par le commandement du Sauveur et formés par l'enseignement d'un Dieu, nous osons dire :}

\versio{%
Pater noster, qui es in cælis : sanctificétur nomen tuum ; advéniat regnum tuum~; fiat volúntas tua sicut in cælo et in terra.
Panem nostrum quotidiánum da nobis hódie ; et dimítte nobis débita nostra, sicut et nos dimíttimus debitóribus nostris. Et ne nos indúcas in tentatiónem.}{%
Notre Père, qui êtes aux cieux, que votre nom soit sanctifié, que votre règne arrive, que votre volonté soit faite sur la terre comme au ciel. Donnez-nous aujourd'hui notre pain de chaque jour ; pardonnez-nous nos of\-fen\-ses comme nous par\-don\-nons à ceux qui nous ont offensés ; et ne nous laissez pas succomber à la tentation.%
}

\versio{%
℟. Sed líbera nos a malo.}{%
℟. Mais délivrez-nous du mal.}

\smallskip

\versio{%
Amen.}{%
Ainsi soit-il.}

\versio{%
Líbera nos, quǽsumus, Dómine, ab ómnibus malis, prætéritis, præséntibus et futúris : et intercedénte beáta, et gloriósa semper Vírgine Dei Genetríce María, cum beátis Apóstolis tuis Petro et Paulo, atque Andréa, et ómnibus Sanctis, da ✠ propítius pacem in diébus nostris : ut, ope misericórdiæ tuæ adiúti, et a peccáto simus semper líberi, et ab omni perturbatióne secúri. Per eúndem Dóminum nostrum Iesum Christum Fílium tuum, qui tecum vivit et regnat in unitáte Spíritus Sancti Deus.}{%
Délivrez-nous, Seigneur, de tous les maux passés, présents et à venir, et, par l'intercession de la bienheureuse et glorieuse Marie, Mère de Dieu, toujours vierge, de vos bienheureux Apôtres Pierre et Paul et André, et de tous les saints, daignez ✠ nous accorder la paix en notre temps ; qu'avec le soutien de votre miséricorde, nous soyons à jamais délivrés du péché et préservés de toute sorte de troubles. Par Notre Seigneur Jésus-Christ votre Fils, qui, étant Dieu, vit et règne avec vous en l'unité du Saint-Esprit,}

\rubrica{À la fin de cette prière, le prêtre brise l'hostie. Cette fraction de l'hostie signifie que l'unique et indivisible Corps de Jésus-Christ (\emph{signi tantum fit fractura} : "il n'est brisé que dans le signe") est distribué aux fidèles afin qu'ils ne forment qu'un seul corps en Jésus-Christ.}

\versio{%
℣. Per ómnia sǽcula sæculórum.}{%
℣. Dans tous les siècles des siècles.}

\versio{%
℟. Amen.}{%
℟. Ainsi soit-il.}
%
\versio{%
℣. Pax ✠ Dómini sit ✠ semper vobís ✠ cum.}{%
℣. La paix ✠ du Seigneur soit ✠ toujours avec ✠ vous.}
%
\versio{%
℟. Et cum spíritu tuo.}{%
℟. Et avec votre esprit.}

\rubrica{Le mélange d'une parcelle de l'hostie dans le précieux Sang signifie l'unité de tous dans le même sacrifice.}

\versio{%
Hæc commíxtio, et consecrátio Córporis et Sánguinis Dómini nostri Iesu Christi, fiat accipiéntibus nobis in vitam ætérnam. Amen.}{%
Que ce mélange sacramentel du Corps et du Sang de Notre Seigneur Jésus-Christ, que nous allons recevoir, nous serve pour la vie éternelle. Ainsi soit-il.}


\needspace{6\baselineskip}
\subsection*{%
Agnus Dei%
}

\rubrica{Les Hébreux immolaient un agneau en mémoire de leur délivrance de la captivité d'Égypte. Cet agneau était la figure du véritable Agneau immolé pour nous délivrer de la captivité du péché et du démon. Selon les paroles de S.~Jean-Baptiste, il est l'\emph{Agneau de Dieu qui enlève le péché du monde.}}

\versio{%
Agnus Dei, qui tollis peccáta mundi, miserére nobis. \emph{bis}%
}{%
Agneau de Dieu, qui ôtez les péchés du monde, ayez pitié de nous. \emph{bis}%
}%

\versio{%
Agnus Dei, qui tollis peccáta mundi, dona nobis pacem.%
}{%
Agneau de Dieu, qui ôtez les péchés du monde, donez-nous la paix.%
}%


\rubrica{Aux messes solennelles, l'oraison suivante pour la paix est suivie du baiser de paix.}

\versio{%
Dómine Iesu Christe, qui dixísti Apóstolis tuis : Pacem relínquo vobis, pacem meam do vobis : ne respícias peccáta mea, sed fidem Ecclésiæ tuæ ; eámque secúndum voluntátem tuam pacificáre et coadunáre dignéris : qui vivis et regnas Deus per ómnia sǽcula sæculórum. Amen.}{%
Seigneur Jésus-Christ, qui avez dit à vos Apôtres : C'est la paix que je vous laisse en héritage, c'est ma paix je vous donne, ne regardez pas mes péchés, mais la foi de votre Église~; daignez, selon votre volonté, lui donner la paix et la rassembler dans l'unité, vous qui, étant Dieu, vivez et régnez dans tous les siècles des siècles. Ainsi soit-il.}

\rubrica{Le baiser de paix, manifestant l'union dans la paix du Christ, est hiérarchiquement transmis de l'autel jusqu'au dernier degré du clergé. \emph{Ne pense pas que ce baiser soit comme ceux qui se donnent sur la place entre amis ordinaires. Il unit les âmes entre elles ; il est réconciliation, et pour cette raison il est saint.} (S.~Cyrille de Jérusalem)}

%\enlargethispage{\baselineskip}

\versio{%
Dómine Iesu Christe, Fili Dei vivi, qui ex voluntáte Patris, cooperánte Spíritu Sancto, per mortem tuam mundum vivificásti : líbera me per hoc sacrosánctum Corpus et Sánguinem tuum ab ómnibus iniquitátibus meis, et univérsis malis : et fac me tuis semper inhærére mandátis, et a te nunquam separári permíttas : Qui cum eódem Deo Patre et Spíritu Sancto vivis et regnas Deus in sǽcula sæculórum. Amen.

%Percéptio Corporis tui, Dómine Iesu Christe, quod ego indígnus súmere præsúmo, non mihi provéniat in iudícium et condemnatiónem : sed pro tua pietáte prosit mihi ad tutaméntum mentis et córporis, et ad medélam percipiéndam : Qui vivis et regnas cum Deo Patre in unitáte Spíritus Sancti Deus, per ómnia sǽcula sæculórum. Amen.%
}{%
Seigneur Jésus-Christ, Fils du Dieu vivant, qui, accomplissant la volonté du Père dans une œuvre commune avec le Saint-Esprit, avez par votre mort donné la vie au monde, délivrez-moi par votre Corps et votre Sang infiniment saints de tous mes péchés et de tout mal. Faites que je reste toujours attaché à vos commandements, et ne permettez pas que je sois jamais séparé de vous, qui, étant Dieu, vivez et régnez avec Dieu le Père et le Saint-Esprit dans les siècles des siècles. Ainsi soit-il.

%Seigneur Jésus-Christ, si j'ose recevoir votre Corps malgré mon indignité, que cela n'entraîne pour moi ni jugement ni condamnation, mais, par votre miséricorde, me serve de sauvegarde et de remède pour l'âme et pour le corps, vous qui, étant Dieu, vivez et régnez avec Dieu le Père en l'unité du Saint-Esprit, dans tous les siècles des siècles. Ainsi soit-il.%
}

\versio{%
Percéptio Corporis tui, Dómine Iesu Christe, quod ego indígnus súmere præsúmo, non mihi provéniat in iudícium et condemnatiónem : sed pro tua pietáte prosit mihi ad tutaméntum mentis et córporis, et ad medélam percipiéndam : Qui vivis et regnas cum Deo Patre in unitáte Spíritus Sancti Deus, per ómnia sǽcula sæculórum. Amen.\looseness=-1%
}{%
Seigneur Jésus-Christ, si j'ose recevoir votre Corps malgré mon indignité, que cela n'entraîne pour moi ni jugement ni condamnation, mais, par votre miséricorde, me serve de sauvegarde et de remède pour l'âme et pour le corps, vous qui, étant Dieu, vivez et régnez avec Dieu le Père en l'unité du Saint-Esprit, dans tous les siècles des siècles. Ainsi soit-il.}


\subsection*{Communion du prêtre}

\versio{%
Panem cæléstem accípiam, et nomen Dómini invocábo.}{%
Je prendrai le pain du ciel et j'invoquerai le nom du Seigneur.}

\versio{%
Dómine, non sum dignus, ut intres sub tectum meum : sed tantum dic verbo, et sanábitur ánima mea. (ter)}{%
Seigneur, je ne suis pas digne que vous entriez sous mon toit ; mais dites seulement une parole, et mon âme sera guérie. (ter)}

\versio{%
Corpus Dómini ✠ nostri Iesu Christi custódiat ánimam meam in vitam ætérnam. Amen.}{%
Que le Corps de notre ✠ Seigneur Jésus-Christ garde mon âme pour la vie éternelle. Ainsi soit-il.}

\versio{%
Quid retríbuam Dómino pro ómnibus quæ retríbuit mihi ? Cálicem salutáris accípiam, et nomen Dómini invocábo. Laudans invocábo Dóminum, et ab inimícis meis salvus ero.}{%
Que rendrai-je au Seigneur pour tous ses bienfaits ? Je prendrai le calice du salut et j'invoquerai le nom du Seigneur. Je louerai le Seigneur en l'invoquant et je serai délivré de mes ennemis.}

\versio{%
Sanguis Dómini ✠ nostri Iesu Christi custódiat ánimam meam in vitam ætérnam. Amen.}{%
Que le Sang de notre ✠ Seigneur Jésus-Christ garde mon âme pour la vie éternelle. Ainsi soit-il.}

\subsection*{Communion des fidèles}

\rubrica{Après avoir communié, le prêtre se retourne vers les fidèles en leur présentant le saint Corps du Christ. \emph{En offrant Jésus-Christ à nos yeux, le prêtre nous révèle comment le Christ lui-même est sorti de son mystérieux sanctuaire divin pour prendre par amour de l'homme la nature de l'homme, pour s'incarner totalement sans se mélanger d'aucune façon.} (Denys l'Aréopagite)}

\versio{%
Ecce Agnus Dei, ecce qui tollis peccáta mundi.}{%
Voici l'Agneau de Dieu, voici celui qui enlève les péchés du monde.}

\rubrica{Nous répondons par les paroles du bienheureux centurion. Cette fois ce n'est plus la guérison d'un serviteur que nous demandons, mais celle de notre âme.}
%
\versio{%
Dómine, non sum dignus, ut intres sub tectum meum : sed tantum dic verbo, et sanábitur ánima mea. \emph{ter}}{%
Seigneur, je ne suis pas digne que vous entriez sous mon toit ; mais dites seulement une parole, et mon âme sera guérie. \emph{ter}}

\rubrica{Aux premiers siècles le diacre invitait ceux qui ne communiaient pas à se retirer : \emph{les choses saintes aux saints, hors d'ici les impurs}, ou encore : \emph{que celui qui ne communie pas se retire}. Aujourd'hui l'Église est plus indulgente mais les conditions pour communier sont les mêmes : \textbf{être baptisé et catholique~; ne pas avoir de péché mortel sur la conscience, ce qui suppose notamment l'observation des lois de l'Église au sujet du mariage~; avoir observé le jeûne eucharistique.} La discipline actuelle est d'une heure de jeûne avant la communion. Nous conseillons cependant de s'en tenir à la discipline d'avant le Concile Vatican II : trois heures pour la nourriture solide, une heure pour le liquide non alcoolisé. La discipline antique (en vigueur jusqu'au milieu du XX\ieme{} siècle) exigeait d'être à jeun depuis minuit.\\
La sainte communion est reçue à genoux et directement dans la bouche.}

\needspace{3\baselineskip}
\rubrica{En donnant la sainte Eucharistie, le prêtre dit :}

\versio{%
Corpus Dómini nostri Iesu Christi custódiat ánimam tuam in vitam ætérnam. Amen.}{%
Que le Corps de Notre Seigneur Jésus-Christ garde votre âme pour la vie éternelle. Ainsi soit-il.}

\rubrica{%
On ne répond rien.}

\rubrica{%
Jésus-Christ est le Bon Pasteur qui donne sa vie pour ses brebis. \emph{Je suis venu pour qu'ils aient la vie et qu'ils l'aient en abondance}. Il nous communique cette vie dans la sainte Eucharistie.%
}

\rubrica{%
\emph{Ce qui te paraît du pain n'est pas du pain, bien qu'il soit tel pour le goût, mais le Corps du Christ. Fortifie donc ton cœur en le prenant comme nourriture spirituelle qui réjouit le visage de ton âme. Et puisses-tu, ce visage découvert en une conscience pure, réfléchir comme un miroir la gloire du Seigneur, et marcher de gloire en gloire dans le Christ Jésus notre Seigneur.} (S.~Cyrille de Jérusalem)%
}

%\vfill

\subsection*{Chant de communion}

\rubrica{Une antienne, alternée avec des versets de psaume, est chantée pendant la communion : \emph{Si nous participons tous à un seul pain, nous devenons tous un seul corps.} (I~Corinthiens 10,17)%
}

\rubrica{%
Par la sainte eucharistie Jésus-Christ nous transforme en lui-même et nous fait communier à sa Passion, à sa Résurrection et à son Ascension. C'est la consommation du Sacrifice en tous ses membres. \emph{Nous sommes tous dans le Père, dans le Fils, dans le Saint-Esprit; un par l'amour et la concorde, avec Dieu et entre nous, par la communauté du saint Corps du Christ et l'unique Esprit-Saint.} \mbox{(S.~Cyrille d'Alexandrie)}%
}

\rubrica{Voir au propre du jour.}

\needspace{5\baselineskip}
\subsection*{Ablutions}

\versio{%
Quod ore súmpsimus, Dómine, pura mente capiámus : et de múnere temporáli fiat nobis remédium sempitérnum.}{%
Ce que notre bouche a reçu, Seigneur, que notre âme l'accueille avec pureté, et que le don qui nous est fait en cette vie nous soit un remède pour la vie éternelle.}

\versio{%
Corpus tuum, Dómine, quod sumpsi, et Sanguis, quem potávi, adhǽreat viscéribus meis : et præsta ; ut in me non remáneat scélerum mácula, quem pura et sancta refecérunt sacraménta : Qui vivis et regnas in sǽcula sæculórum. Amen.}{%
Votre Corps que j'ai mangé et votre Sang que j'ai bu, Seigneur, qu'ils adhèrent à mes entrailles ; et maintenant que je suis restauré par ce sacrement si pur et si saint, faites que le péché ne laisse en moi aucune tache ; vous qui vivez et régnez dans les siècles des siècles. Ainsi soit-il.}


\subsection*{%
Postcommunion%
}

\versio{%
℣. Dóminus vobíscum.}{%
℣. Le Seigneur soit avec vous.}

\versio{%
℟. Et cum spíritu tuo.}{%
℟. Et avec votre esprit.}

\versio{%
Orémus.}{%
Prions.}

\rubrica{Voir au propre du jour. À la fin, on répond :}

\versio{%
℟. Amen.}{%
℟. Ainsi soit-il.}
\pagebreak[3]


\subsection*{Fin de la messe}

\versio{%
℣. Dóminus vobíscum.}{%
℣. Le Seigneur soit avec vous.}

\versio{%
℟. Et cum spíritu tuo.}{%
℟. Et avec votre esprit.}

\versio{%
℣. Ite, missa est.%
}{%
℣. Allez, la messe est dite.%
}

\versio{%
℟. Deo grátias.%
}{%
℟. Rendons grâces à Dieu.%
}%


\needspace{3\baselineskip}
\rubrica{%
Le Prêtre, incliné, récite à voix basse la prière suivante :}

\versio{%
Pláceat tibi, sancta Trínitas, obséquium servitútis meæ : et præsta ; ut sacrifícium, quod óculis tuæ maiestátis indígnus óbtuli, tibi sit acceptábile, mihíque et ómnibus, pro quibus illud óbtuli, sit, te miseránte, propitiábile. Per Christum Dóminum nostrum. Amen.}{%
Agréez, Trinité Sainte, l'hommage de votre serviteur : ce sacrifice que malgré mon indignité j'ai présenté aux regards de votre Majesté, rendez-le digne de vous plaire et capable, par l'effet de votre miséricorde, d'attirer votre faveur sur moi-même et sur tous ceux pour qui je l'ai offert. Par le Christ notre Seigneur. Ainsi soit-il.}

\needspace{3\baselineskip}
\rubrica{%
Puis il donne la bénédiction finale :}

\versio{%
Benedícat vos omnípotens Deus, Pater, et ✠ Fílius, et Spíritus Sanctus.}{%
Que Dieu tout-puissant vous bénisse, le Père, ✠ le Fils et le Saint-Esprit.}


\subsection*{Dernier Évangile}

\rubrica{Le dernier Évangile rappelle le mystère de l'Incarnation et le but du Sacrifice qui vient d'être célébré : devenir des enfants de Dieu.}
%
\versio{%
℣. Dóminus vobíscum.}{%
℣. Le Seigneur soit avec vous.}
%
\versio{%
℟. Et cum spíritu tuo.}{%
℟. Et avec votre esprit.}
%
\versio{%
℣. ✠ Inítium sancti Evangélii secúndum Ioánnem.}{%
℣. ✠ Commencement du saint Évangile selon saint Jean.}
%
\versio{%
℟. Glória tibi Dómine.}{%
℟. Gloire à vous, Seigneur.}
%
\versio{%
In princípio erat Verbum, et Verbum erat apud Deum, et Deus erat Verbum. Hoc erat in princípio apud Deum. Omnia per ipsum facta sunt : et sine ipso factum est nihil, quod factum est : in ipso vita erat, et vita erat lux hóminum : et lux in ténebris lucet, et ténebræ eam non comprehendérunt. Fuit homo missus a Deo, cui nomen erat Ioánnes. Hic venit in testimónium, ut testimónium perhibéret de lúmine, ut omnes créderent per ilIum. Non erat ille lux, sed ut testimónium perhibéret de lúmine. Erat lux vera, quæ illúminat omnem hóminem veniéntem in hunc mundum. In mundo erat, et mundus per ipsum factus est, et mundus eum non cognóvit. In própria venit, et sui eum non recéperunt. Quotquot autem recepérunt eum, dedit eis potestátem fílios Dei fíeri, his, qui credunt in nómine ejus : qui non ex sanguínibus, neque ex voluntáte carnis, neque ex voluntáte viri, sed ex Deo nati sunt. Et Verbum caro factum est, et habitávit in nobis : et vídimus glóriam eius, glóriam quasi Unigéniti a Patre, plenum grátiæ et veritátis.}{%\widowpenalty=10000%
Au commencement était le Verbe, et le Verbe était auprès de Dieu, et le Verbe était Dieu. Il était auprès de Dieu au commencement. Tout a été fait par lui, et rien de ce qui a été fait n'a été fait sans lui. En lui était la vie, et la vie était la lumière des hommes. La lumière luit dans les ténèbres, et les ténèbres ne l'ont pas accueillie. Il y eut un homme envoyé par Dieu, du nom de Jean. Il vint comme témoin, pour rendre témoignage à la lumière, afin que tous croient par lui. Il n'était pas lui-même la lumière, mais il venait seulement rendre témoignage à la lumière. Le Verbe était la vraie lumière qui éclaire tout homme venant en ce monde. Il était dans le monde, et le monde a été fait par lui, et le monde ne l'a pas connu. Il est venu chez lui, et les siens ne l'ont pas reçu. Mais à tous ceux qui l'ont reçu, il a donné le pouvoir de devenir enfants de Dieu, à ceux qui croient en son Nom, qui ne sont pas nés du sang, ni de la volonté de la chair, ni de la volonté de l'homme, mais de Dieu. Et le Verbe s'est fait chair, et il a habité parmi nous. Et nous avons vu sa gloire, gloire que le Père donne à son Fils unique, plein de grâce et de vérité.}

\rubrica{%
À la messe lue, on répond :}

\versio{%
℟. Deo grátias.}{%
℟. Rendons grâces à Dieu.}%


%%%%%%%%%%%%%%%%%%%%%%%%%%%%%%%%%%%%%%%%%%%%%%%%%%%%%%%%%%%%%%%%%%%%%%%%%%%%%%%%

\addchap{Sacrement de pénitence}

Le sacrement de pénitence ou confession fut institué par Notre Seigneur Jésus-Christ, pour effacer les péchés commis après le baptême. Notre-Seigneur l’a transmis aux Apôtres lorsqu’il leur a dit : « Recevez le Saint-Esprit ; les péchés seront remis à ceux à qui vous les remettrez et ils seront retenus à ceux à qui vous les retiendrez. »


\addsec{Pour les enfants}

Pour bien recevoir le sacrement de pénitence, il faut connaître ses péchés, en avoir la contrition, les accuser, et après en avoir reçu l’absolution, faire la pénitence imposée par le prêtre.

\subsection*{Qu’est-ce que se confesser ?}

C’est dire ses péchés à un prêtre pour en recevoir l’absolution.

\subsection*{Comment se confesser ?}

Deux choses à faire :
\begin{itemize}
\item retrouver nos péchés ;
\item les regretter.
\end{itemize}

\emph{Pour les chercher}, fermons les yeux et faisons l’examen de conscience. Nous pouvons nous aider de l’examen de conscience qui suit.
\pagebreak[3]

\emph{Pour les regretter}, disons lentement :\\
Esprit-Saint, qui êtes la lumière de nos cœurs, éclairez ma conscience, montrez-moi mes péchés, faites que je les voie, comme je les verrai à l’heure de mon jugement et comme les voyait Jésus, quand il mourait pour les réparer. Montrez-moi les défauts qui m’ont poussé à les commettre, pour que je les combatte. Faites que je sois bien décidé à suivre les conseils de votre prêtre, pour que votre grâce en moi rencontre moins d’obstacles, et puisse me guérir de mes mauvais penchants. Ainsi-soit-il !

\subsection*{Examen de conscience}

\minisec{Confession précédente}

\begin{itemize}
\item Combien y a-t-il de temps que je ne me suis pas confessé ?
\item Ai-je bien dit tous mes péchés ?
  \begin{itemize}
  \item N’ai-je pas caché volontairement des péchés graves ? (Si oui, il vous faut absolument vous en accuser, car non seulement aucun de vos péchés accusés n’a été pardonné, mais vous avez ajouté un autre péché très grave : un sacrilège).
  \item N’ai-je pas oublié des péchés graves ? (Si oui, votre dernière confession a été bonne quand même ; mais il faut que vous les accusiez maintenant).
  \end{itemize}
\item Me suis-je mal préparé à ma dernière confession ?
\item N’ai-je pas manqué de contrition, c’est-à-dire de vrai repentir de mes fautes ? (pour avoir un vrai repentir, il faut être décidé à faire tout son possible pour ne pas recommencer).
\item Ai-je fait ma pénitence ?
\item Quelle résolution avais-je prise lors de ma dernière confession ? Est-ce que je l’ai tenue ?
\end{itemize}

\needspace{3\baselineskip}
\minisec{Commandements de Dieu}

\commandement{1. Tu adoreras Dieu seul et tu l'aimeras plus que tout.}

\begin{itemize}
\item Ai-je manqué mes prières ? du matin ? du soir ? les ai-je mal faites ?
\item Me suis-je mal tenu à l’église ? Ai-je dissipé les autres ?
\item Ai-je eu honte de paraître chrétien ?
\item Ai-je tenu des conversations contre la religion ?
\end{itemize}

\commandement{2. Tu ne prononceras le nom de Dieu qu’avec respect.}
\begin{itemize}
\item Ai-je dit des gros mots ? Ai-je dit des jurons ?
\item Ai-je fait des serments pour des riens ?
\end{itemize}

\commandement{3. Tu sanctifieras le jour du Seigneur.}
\begin{itemize}
\item Ai-je manqué par ma faute, la messe le dimanche ou les fêtes d’obligation ? Combien de fois ?
\item Suis-je arrivé en retard ? A quel moment ?
\end{itemize}

\commandement{4. Tu honoreras ton père et ta mère.}
\begin{itemize}
\item Ai-je désobéi à mes parents ?
\item Leur ai-je mal répondu ? Me suis-je moqué d’eux ?
\item Ai-je fait la tête ? Ai-je fait du mauvais esprit ?
\end{itemize}

\commandement{5. Tu ne tueras pas.}
\begin{itemize}
\item Me suis-je disputé avec les autres ?
\item Ai-je gardé rancune ? Ai-je cherché à me venger ?
\item Ai-je donné le mauvais exemple ? ou entraîné d’autres à pécher ?
\end{itemize}

\commandement{6. Tu ne feras pas d'impureté.\\
9. Tu n'auras pas de désir impur volontaire.}
\begin{itemize}
\item Ai-je regardé des images mauvaises, impures ? Ai-je cherché exprès des journaux impurs ?
\item Ai-je vu des spectacles mauvais (à la télévision par exemple) ?
\item Ai-je accepté des pensées impures ? des désirs impurs ?
\item Ai-je participé à de mauvaises conversations ?
\item Ai-je fait des actions impures ? seul ? avec d’autres ?
\end{itemize}

\commandement{7. Tu ne voleras pas.\\
10. Tu ne désireras pas injustement le bien des autres.}
\begin{itemize}
\item Ai-je pris ou recherché à prendre quelque chose qui n’était pas à moi (des gourmandises, de l’argent) ?
\item Ai-je abîmé exprès ce qui ne m’appartenait pas ?
\item Ai-je triché au jeu ? Ai-je copié en classe, à un examen, à une composition, à un devoir ?
\end{itemize}

\commandement{8. Tu ne mentiras pas.}
\begin{itemize}
\item Ai-je menti ? (pour m’amuser, pour me vanter, pour ne pas être puni, pour tromper).
\item Ai-je dit du mal des autres ? Ai-je cherché à faire punir les autres ?
\item  Ai-je pensé sans raison suffisante, du mal des autres ?
\end{itemize}


\needspace{3\baselineskip}
\minisec{Commandements de l'Église}

\begin{itemize}
\item Me suis-je bien préparé à ma dernière
 communion ?
\item Ai-je communié sans être à jeun ?
\item Ai-je communié avec des péchés graves sur la conscience ?
\item Ai-je mangé de la viande les jours défendus ?
\end{itemize}


\needspace{3\baselineskip}
\minisec{Péchés capitaux}

\begin{itemize}
\item Ai-je été orgueilleux ? Ai-je refusé de reconnaître mes torts ? Ai-je rabaissé les autres en pensée ? Me suis-je vanté ? Me suis-je vexé pour rien ?
\item Ai-je été gourmand : en étant difficile ? en mangeant trop de friandises ? en
mangeant et en buvant avec excès ? Ai-je fumé en cachette ?
\item Ai-je été avare ? Ai-je refusé de prêter mes affaires ?
\item Ai-je été jaloux ?
\item Me suis-je mis en colère ? Ai-je eu mauvais caractère, rendant la vie pénible
autour de moi ? Ai-je été impatient ? Ai-je fait mettre exprès les autres en colère ?
\item Ai-je été paresseux ? pour me lever ? pour prier ? pour communier ? à l’école ? au catéchisme ? pour faire mes devoirs ? Ai-je manqué par ma faute l’école ou le catéchisme ?
\end{itemize}

Mon défaut dominant est : …

Je prends la résolution de : …


\addsec{Pour les adultes}

Le sacrement de Pénitence, appelé aussi confession, est le sacrement institué par Jésus-Christ pour remettre les péchés commis après le baptême. Le ministre de ce sacrement est le prêtre. Tenant la place de Jésus-Christ et recevant la confidence de nos péchés même les plus cachés, le prêtre est tenu à un secret absolu sur tout ce qu’il a entendu. C’est le secret de confession. Même sous la menace de mort ou de torture, il ne peut rien dire et rien révéler. Nous pouvons donc lui parler en toute confiance et sans crainte.

Ce sacrement exige quatre conditions :
\begin{enumerate}
\item la connaissance de nos péchés ;
\item la contrition de nos péchés ;
\item la confession de nos péchés au prêtre suivie de l’absolution ;
\item la satisfaction pour nos péchés.
\end{enumerate}

Les parties du sacrement sont :
\begin{itemize}
\item la contrition : c’est un acte de volonté, une douleur de l’âme et l’horreur du péché commis, et la résolution de ne plus pécher à l’avenir ;
\item la confession : elle consiste dans l’accusation détaillée de nos péchés faite au confesseur pour en avoir l’absolution et la pénitence ;
\item l'absolution : c’est la phrase que le prêtre prononce au nom de Jésus-Christ, pour remettre les péchés au pénitent ;
\item la satisfaction : ou pénitence sacramentelle, c’est la prière ou la bonne œuvre imposée par le confesseur pour le châtiment et la correction du pêcheur, et l’escompte de la peine temporelle méritée en péchant.
\end{itemize}

Les effets de la confession bien faite : le sacrement de pénitence :
\begin{itemize}
\item donne la grâce sanctifiante avec laquelle les péchés mortels, et aussi les péchés véniels, confessés et que l'on regrette, nous sont remis ;
\item commue la peine éternelle en temporelle ; celle-ci est diminuée dans la mesure de la contrition ;
\item rend les mérites des bonnes œuvres faites avant de commettre le péché
mortel ;
\item donne à l’âme les secours nécessaires pour ne pas retomber dans le péché et redonner la paix à la conscience.
\end{itemize}

Pour préparer une bonne confession :

dans la confession il faut accuser au moins tous les péchés mortels, pas encore bien confessés (dans une bonne confession) et ceux que l’on se rappelle. Indiquer, dans la mesure du possible, leur espèce et leur nombre.

Pour cela on demande à Dieu la grâce de bien connaître ses fautes, et on s’examine sur les dix commandements et les préceptes de l’Église, sur les péchés capitaux et les devoirs de notre état.

N.B. 1. Pour reconnaître un péché mortel (c’est-à-dire qui donne la mort surnaturelle à l’âme), il faut trois choses :
\begin{itemize}
\item la gravité de la matière ;
\item la pleine advertance (c’est-à-dire la pleine connaissance) ;
\item le plein consentement.
\end{itemize}

N.B. 2. L’accusation de l’espèce et du nombre est de rigueur pour les désirs, au moins approximativement s'il n'est pas possible de se souvenir du nombre exact.


\subsection*{Examen de conscience}

\minisec{Commandements de Dieu}

\commandement{1. Tu adoreras Dieu seul et tu l'aimeras plus que tout.}

Manqué à mes prières, les ai mal faites.
Craint de me montrer chrétien, par respect humain. Négligé de m’instruire des
vérités de la religion, doutes volontaires.
Lu des livres, des journaux impies. Parlé,
agi contre la religion. Murmuré contre
Dieu et sa providence. Appartenu à des
sociétés impies (franc-maçonnerie, communisme, sectes hérétiques, etc.) Pratiqué
des superstitions, consulté les cartes et les
devins. Avoir tenté Dieu.

Péchés contre la foi : refuser d’admettre
une ou plusieurs vérités révélées de Dieu.
Péchés contre l’espérance : manquer de
confiance en la bonté et providence de
Dieu. Prétendre qu’il soit impossible de
vivre en vrai chrétien quoiqu’on en demande la grâce. Pécher par présomption
en abusant de la bonté de Dieu.

Péchés contre la charité : refuser d’aimer
Dieu par-dessus tout. Passer des semaines
et des mois sans faire le plus petit acte
d’amour de Dieu. Indifférence religieuse.
Sacrilèges en profanant les choses saintes,
en particulier confessions et communions
sacrilèges.

Charité envers le prochain : refuser de
voir Dieu dans nos frères, d’aimer Dieu
dans le prochain. Mépriser, détester, se
moquer du prochain.

\commandement{2. Tu ne prononceras le nom de Dieu qu’avec respect.}

Fait des serments faux ou inutiles — Imprécations contre moi-même ou contre
d’autres — Manqué de respect à l’égard
du nom de Dieu ou des saints — Blasphémé en murmurant contre Dieu dans
les épreuves — Manqué à des vœux.

\commandement{3. Tu sanctifieras le jour du Seigneur.}

À ce commandement se rapportent les 1\ier\ et 2\ieme\ commandements de l’Église.
Manqué à la messe le dimanche par ma faute, arrivé
en retard, assisté sans respect. Travaillé
ou fait travailler sans nécessité et sans
permission. Avoir profané
cette journée par des réunions ou amusements dangereux pour la foi ou les
mœurs.

\commandement{4. Tu honoreras ton père et ta mère.}

Enfants : manqué de respect. Désobéi.
Causé du chagrin à mes parents. Négligé
de les assister. N’avoir pas tenu compte de
leurs sages avis.

Parents : ai-je pensé à donner ou procurer
une instruction religieuse à nos enfants ?
Les ai-je fait prier ? Ai-je choisi pour eux
l’école la plus sûre ? Veillé sur eux avec
diligence ? Les ai-je conseillés, repris,
corrigés ?

Époux : l’amour entre les conjoints est-il
vraiment patient, empressé, prêt à tout ?
Manque de support mutuel. Avoir critiqué
son conjoint devant les enfants.

Inférieurs (employés, ouvriers, soldats) :
Manque de respect, d’obéissance à mes
supérieurs — Fait du tort par des critiques
— Négligé mon service. Commis des abus
de confiance.

Supérieurs : manqué à la justice commutative, en ne donnant pas ce qui était dû, à
la justice sociale. Manqué à la charité, en
ne procurant pas les secours nécessaires
— N’avoir pas traité ses employés avec
bonté, équité, charité

\commandement{5. Tu ne tueras pas.}

M’être mis en colère. Voulu me venger.
Souhaité du mal — Haines, rancunes, refuser de pardonner — Ai injurié, blessé
— Impatiences ­ — Mauvais conseils —
Scandalisé par paroles... actions — Infanticide — Avortement — Euthanasie —
Transgressions graves au code de la route,
de façon volontaire (même s’il n’est rien
arrivé).

\commandement{6. Tu ne feras pas d'impureté.\\
9. Tu n'auras pas de désir impur volontaire.}

M’être arrêté volontairement à des pensées, à des désirs contraires à la pureté
— Conversations et chansons légères ou
déshonnêtes, vêtements indécents — Télévision, radio (mauvaises émissions),
gravures, livres, journaux mauvais — Regards, familiarités coupables — Actions
déshonnêtes, seul... avec d’autres —
Liaisons ou fréquentations coupables —
Fraudes dans l’usage du mariage — Refus
du dû conjugal — Péchés entre fiancés.

\commandement{7. Tu ne voleras pas.\\
10. Tu ne désireras pas injustement le bien des autres.}

Désiré prendre le bien d’autrui ­ — Commis ou aidé à commettre des injustices,
des fraudes, des vols — Causé du dommage — Pas restitué — Pas payé mes dettes — Fait tort dans les ventes, contrats,
transactions, etc.

\commandement{8. Tu ne mentiras pas.}

Ai menti — Jugé témérairement — Dit
du mal du prochain — Calomnié — Faux
témoignage — Violé des secrets (lu une
lettre). Fait ou diffusé des soupçons.


\minisec{Commandements de l'Église}

\commandement{1. Les fêtes tu sanctifieras qui te sont de commandement.\\
2. Les dimanches Messe tu entendras, et les fêtes pareillement.}

Voir au 3\ieme\ commandement de Dieu.

\commandement{3. Tous tes péchés confesseras, à tout le moins une fois l’an.\\
4. Ton créateur tu recevras, au moins à Pâques humblement.}

Être resté plus d'un an sans confesser un péché grave. Ne pas avoir communié, au moins à Pâques. Avoir fait des communions sacrilèges.

\commandement{5. Vigiles, pénitence feras − Carême et Quatre-Temps également.\\
6. De la viande ne mangeras les jours défendus mêmement.}

Avoir sans raison légitime et sans permission : manqué au jeûne - mangé de la viande les jours défendus - manqué au
jeûne eucharistique.

\commandement{7. Loi de l'Index}

Avoir lu ou conservé des livres, revues ou journaux défendus expressément par l’Église ou contre la foi et les mœurs.


\minisec{Péchés capitaux}

\commandement{Orgueil}

Agi par orgueil - Dépenses et luxes exagérés - Méprisé les autres - M’être complu
dans des pensées de vanité - Susceptibilités - Être esclave du « qu’en dira-t-on ? »
et la mode.

\commandement{Avarice}

M’être trop attaché à l’argent - N’avoir
pas fait l’aumône selon mes moyens - Jeux
d’argent (Voir aux 7\ieme\ et 10\ieme\ commandements
de Dieu).

\commandement{Luxure} : voir aux 6\ieme\ et 9\ieme\ commandements de Dieu.

\commandement{Envie}

Avoir entretenu des sentiments de jalousie - Cherché à nuire aux autres par envie
- M’être réjoui du mal ou attristé du bien
d’autrui.

\commandement{Gourmandise}

Excès dans le manger, dans le boire.
Ivresse : combien de fois ? Usage de stupéfiants.

\commandement{Colère} : voir au 5\ieme\ commandement de Dieu.

\commandement{Paresse}

Au lever. Dans le travail. Dans les devoirs religieux.


%%%%%%%%%%%%%%%%%%%%%%%%%%%%%%%%%%%%%%%%%%%%%%%%%%%%%%%%%%%%%%%%%%%%%%%%%%%%%%%%

\addchap{Les 18 apparitions}

\newcommand{\apparition}[3]{%
#1
  & \parbox[c][\height + 2ex][c]{\linewidth}{\centering #2\par}
  & \parbox[c][\height + 2ex][c]{\linewidth}{#3}\\
}

\definecolor{fond}{gray}{.92}

{\rowcolors{1}{fond}{}

\noindent\begin{tabularx}{\textwidth}{|l|p{5em}|X|}
\hline
\apparition{1}{Jeudi 11 février\\ à 11 heures\\ (prologue)}{Contemplation silencieuse.\\
« La dame prit le chapelet qu’elle tenait entre ses mains et elle fit le signe de croix.
Alors j’ai essayé une seconde fois de le faire et je pus. Aussitôt que j’eus fait le signe
de croix le grand saisissement que j’éprouvais disparut. Je me mis à genoux. J’ai
passé mon chapelet en présence de cette belle dame... »}
\hline
\apparition{2}{Dimanche 14 février\\ avant les vêpres}{Contemplation silencieuse.\\
« Alors je me mis à lui jeter de l’eau bénite tout en lui disant : si elle venait de la part
de Dieu de rester, sinon de s’en aller »}
\hline
\apparition{3}{Jeudi 18 février,\\ lendemain des Cendres}{%
\emph{« Voulez-vous me faire la grâce de venir ici 15 jours de suite ? »} Je lui répondais que oui.\\
\emph{« Je ne vous promets pas de vous rendre heureuse en ce monde, mais dans l’autre. »}%
}
\hline
\apparition{4}{Vendredi 19 février}{%
Contemplation silencieuse.\\
Pour la première fois, Bernadette porte un cierge bénit (jusqu’au 25 mars). La
Dame paraît et, à sa vue, Bernadette demeure en extase. Elle raconte que
l’une des voix sinistres a crié : « Sauve-toi ! Sauve-toi ! » Mais la Dame a levé
la tête et froncé les sourcils et les voix se sont évanouies.}
\hline
\apparition{5}{Samedi 20 février\\ à 6 heures}{%
Contemplation silencieuse.\\
Elle a eu la bonté de lui apprendre, mot par mot et phrase par phrase, une
prière pour elle seule.}
\hline
\apparition{6}{Dimanche 21 février}{%
Contemplation silencieuse.}
\hline
\apparition{}{Lundi 22 février}{%
Pas d’apparition. « Je ne sais pas en quoi j’ai manqué à cette dame puisque elle
n’est pas apparue ! »}
\hline
\apparition{7}{Mardi 23 février}{%
Contemplation silencieuse.\\
Bernadette dira qu’à cette occasion lui ont été confiés « trois secrets personnels
avec ordre de ne les révéler à personne ».}
\hline
\apparition{8}{Mercredi 24 février}{%
\emph{« Pénitence ! Pénitence ! Pénitence ! »}\\
\emph{« Vous prierez Dieu pour les pécheurs. »}\\
\emph{« Allez baiser la terre en pénitence pour la conversion des pécheurs. »}%
}
\hline
\apparition{9}{Jeudi 25 février}{%
\emph{« Allez boire à la fontaine et vous y laver. »}
(Découverte de la source).
\emph{« Vous mangerez de cette herbe qui est là. »}%
}
\hline
\apparition{}{Vendredi 26 février}{%
Pas d’apparition.%
}
\hline
\apparition{10}{Samedi 27 février}{%
\emph{« Pénitence ! Pénitence ! Pénitence ! »}\\
\emph{« Vous prierez Dieu pour les pécheurs. »}\\
\emph{« Allez baiser la terre en pénitence pour la conversion des pécheurs. »}%
}
\hline
\apparition{11}{Dimanche 28 février}{%
\emph{« Pénitence ! Pénitence ! Pénitence ! »}\\
\emph{« Vous prierez Dieu pour les pécheurs. »}\\
\emph{« Allez baiser la terre en pénitence pour la conversion des pécheurs. »}\\
Un mot du garde-champêtre, et la foule recueillie, imitant Bernadette, baise la
terre.%
}
\hline
\apparition{12}{Lundi 1\ier mars}{%
Contemplation silencieuse.%
}
\hline
\apparition{13}{Mardi 2 mars}{%
\emph{« Allez dire aux prêtres qu’on vienne ici en procession et
qu’on y bâtisse une chapelle. »}%
}
\hline
\apparition{14}{Mercredi 3 mars\\ à 7h de l'après-midi}{%
La vision n’apparaît pas le matin. En raison de la présence indésirable de
certains pécheurs, la Vierge ne se montre pas à l'heure habituelle.\\
\emph{« Allez dire aux prêtres de faire bâtir ici une chapelle. »}\\
\emph{« Je vous défends de dire cela à personne. »} (Le secret confié).%
}
\hline
\apparition{15}{Jeudi 4 mars}{%
\emph{« Allez dire aux prêtres de faire bâtir ici une chapelle. »}%
}
\hline
\apparition{16}{Jeudi 25 mars}{%
\emph{« Allez dire aux prêtres de faire bâtir ici une chapelle. »}\\
« Madame, voulez-vous avoir la bonté de me dire qui vous êtes, s’il vous plaît ? »
(à 4 reprises)\\
\textbf{« Que soy era Immaculada Conceptiou. »}\\
« Après les quinze jours, je lui ai demandé de nouveau trois fois de suite qui elle
était. Elle souriait toujours. Enfin je m’hasardais une quatrième fois. Alors, tenant
les deux bras pendants elle leva les yeux en regardant le ciel, puis elle me dit, en
joignant les mains à la hauteur de la poitrine qu’elle était l’Immaculée Conception.
Ce sont les dernières paroles qu’elle m’a adressées ».%
}
\hline
\apparition{17}{Mercredi de Pâques,\\ 7 avril, à 5h du matin}{%
Contemplation silencieuse.\\
Le « miracle du cierge ».%
}
\hline
\apparition{18}{Vendredi 16 juillet}{%
Dernière apparition, silencieuse.\\
« Elle m’apparut au lieu ordinaire sans rien me dire… Je ne l’avais jamais vue aussi
belle ». L’apparition terminée, Bernadette se lève avec la même attitude lente
et recueillie. Sur son visage on peut lire une grande joie.%
}
\hline
\end{tabularx}
}

%%%%%%%%%%%%%%%%%%%%%%%%%%%%%%%%%%%%%%%%%%%%%%%%%%%%%%%%%%%%%%%%%%%%%%%%%%%%%%%%
%%%%%%%%%%%%%%%%%%%%%%%%%%%%%%%%%%% Samedi 25 %%%%%%%%%%%%%%%%%%%%%%%%%%%%%%%%%%
%%%%%%%%%%%%%%%%%%%%%%%%%%%%%%%%%%%%%%%%%%%%%%%%%%%%%%%%%%%%%%%%%%%%%%%%%%%%%%%%

\part[Samedi 25 octobre]{%
\bgimage{Vierge}\\
Samedi 25 octobre\\%
\ \\ %
{\large En l'honneur de l'Immaculée Conception\\%
\ \\ %
\emph{À la fin, mon Cœur immaculé triomphera.}%
}%
}

\vspace*{\stretch{1}}
\section*{Programme de la journée}

\begin{tabularx}{\textwidth-\parindent}{l!{:}X}
14h00	& Chapelet. \lieu{Basilique Saint-Pie X}\\
14h30	& Messe de l'Immaculée Conception, célébrée par son Excellence Mgr Alfonso \textsc{de Galarreta}.
	  \lieu{Basilique Saint-Pie X}\linebreak
	  Ministres supérieurs :\hspace*{\stretch{1}}\linebreak
	  − Prêtre-assistant : M. l'abbé Yves \textsc{Le Roux}, directeur du séminaire de Winona ;\hspace*{\stretch{1}}\linebreak
	  − Diacres-assistants : M. l'abbé Pierpaolo \textsc{Petrucci}, supérieur du district d'Italie, et M. l'abbé Philippe \textsc{Brunet}, supérieur de la maison autonome d'Espagne − Portugal.\hspace*{\stretch{1}}\linebreak
	  Service liturgique assuré par le séminaire international St-Thomas d'Aquin, de Winona (États-Unis).\\
16h00	& Chemin de Croix. \lieu{Basilique Saint-Pie X}\\
18h00	& Conférence\footnote{Les conférences seront données en français, italien, espagnol, anglais et allemand.} : \emph{Saint Pie X et la Très Sainte Vierge Marie}.
	  \lieu{Église Sainte-Bernadette}\\
21h15	& Procession aux flambeaux, puis \textbf{renouvellement de la consécration de la FSSPX au Cœur immaculé de Marie} devant la Grotte.
	  \lieu{Au podium, à côté de Sainte-Bernadette}
\end{tabularx}

\medskip
Nuit de prière à la Grotte.

\vfill
Attention :
\begin{itemize}
\item les sanctuaires ferment leurs portes à minuit : passée cette heure, il vous faudra emprunter la porte des lacets, derrière la grotte (voir plan à la fin, n° 5) ;
\item changement d'heure cette nuit : à 3h, il sera 2h.
\end{itemize}
\vspace{\stretch{2}}


%%%%%%%%%%%%%%%%%%%% Messe votive de l'Immaculée Conception %%%%%%%%%%%%%%%%%%%%

\addchap{Messe de l'Immaculée Conception}

\subsection*{Introït}

\vulgo{Je me réjouirai avec effusion dans le Seigneur, et mon âme sera ravie d'allégresse en mon Dieu : car il m'a revêtu des vêtements du salut : et il m'a entouré des ornements de la justice, comme une épouse parée de ses bijoux.
℣. Je vous exalterai, Seigneur, parce que vous m'avez relevé, et que vous n'avez pas réjoui mes ennemis à mon sujet.}

\cantus{Introit}{GaudensGaudebo_Quasi}{Intr.}{3.}

\needspace{5\baselineskip}

{\centering\emph{Kyriale IX}\par}

\subsection*{Collecte}

\oratio{%
Deus, qui per immaculátam Vírginis Conceptiónem dignum Fílio tuo habitáculum præparásti : quǽsumus ; ut, qui ex morte eiúsdem Filii tui prævísa eam ab omni labe præservásti, nos quoque mundos eius intercessióne ad te perveníre concédas. Per eúndem Dóminum nostrum Iesum Christum Fílium tuum, qui tecum vivit et regnat in unitáte Spíritus Sancti Deus, per ómnia sǽcula sæculórum.%
}{%
Ô Dieu, qui, par l'Immaculée Conception de la Vierge, avez préparé à votre Fils une demeure digne de lui, nous vous en supplions, vous qui, en prévision de la mort de ce même Fils, l'avez préservée de toute tache, accordez-nous, par son intercession, d'être purifiés et de parvenir à vous. Par Notre Seigneur Jésus-Christ votre Fils, qui, étant Dieu, vit et règne avec vous en l'unité du Saint-Esprit, dans tous les siècles des siècles.%
}

\subsection*{Épître}

\versio{%
Léctio libri Sapiéntiæ.

Dóminus possedit me in inítio viárum suárum, ántequam quidquam fáceret a princípio. Ab ætérno ordináta sum, et ex antíquis, ántequam terra fíeret. Nondum erant abýssi, et ego iam concépta eram : necdum fontes aquárum erúperant : necdum montes gravi mole constíterant : ante colles ego parturiébar : adhuc terram non fécerat et flúmina et cárdines orbis terræ. Quando præparábat cælos, áderam : quando certa lege et gyro vallábat abýssos : quando .thera firmábat sursum et librábat fontes aquárum : quando circúmdabat mari términum suum et legem ponébat aquis, ne transírent fines suos : quando appendébat fundaménta terræ. Cum eo eram cuncta compónens : et delectábar per síngulos dies, ludens coram eo omni témpore : ludens in orbe terrárum : et delíciæ meæ esse cum filiis hóminum. Nunc ergo, filii, audíte me : Beáti, qui custódiunt vias meas. Audíte disciplínam, et estóte sapiéntes, et nolíte abícere eam. Beátus homo, qui audit me et qui vígilat ad fores meas cotídie, et obsérvat ad postes óstii mei. Qui me invénerit, invéniet vitam et háuriet salútem a Dómino.%
}{%
{\small Lecture du Livre de la Sagesse.

Le Seigneur m'a possédée au commencement de ses voies, avant de faire quoi que ce soit, dès le principe. J'ai été établie dès l'éternité, et dès les temps anciens, avant que la terre fût créée. Les abîmes n'étaient pas encore, et déjà j'étais conçue ; les sources des eaux n'avaient pas encore jailli ; les montagnes ne s'étaient pas encore dressées avec leur pesante masse ; j'étais enfantée avant les collines. Il n'avait pas encore fait la terre, ni les fleuves, ni les bases du globe terrestre. Lorsqu'il préparait les cieux, j'étais là ; lorsqu'il environnait les abîmes de leurs bornes, par une loi inviolable ; lorsqu'il affermissait l'air dans les régions supérieures, et qu'il équilibrait les sources des eaux ; lorsqu'il entourait la mer de ses limites, et qu'il imposait une loi aux eaux, pour qu'elles ne franchissent point leurs bornes, lorsqu'il posait les fondements de la terre, j'étais avec lui, réglant toutes choses, et j'étais chaque jour dans les délices, me jouant sans cesse devant lui, me jouant sur le globe de la terre, et mes délices sont d'être avec les enfants des hommes. Maintenant donc, mes fils, écoutez-moi : Heureux ceux qui gardent mes voies. Écoutez mes instructions et soyez sages, et ne les rejetez pas. Heureux l'homme qui m'écoute, et qui veille tous les jours à ma porte, et qui se tient à la porte de ma maison. Celui qui me trouvera trouvera la vie, et puisera le salut dans le Seigneur.}%
}

\subsection*{Graduel}

\vulgo{Bénie êtes-vous, ô Vierge Marie, par le Seigneur Dieu très-Haut, plus que toutes les femmes sur la terre. ℣. Vous, gloire de Jérusalem ; vous, joie d'Israël ; vous, honneur de notre peuple.}

\cantus{Graduel}{Benedicta}{Grad.}{5.}

\subsection*{Alléluia}

\vulgo{Alléluia, alléluia. ℣. Vous êtes toute belle, ô Marie, et la tache originelle n'est pas en vous. Alléluia.}

\cantus{Alleluia}{TotaPulchraEs}{}{1.}

\subsection*{Évangile}

\versio{%
✠ Sequéntia sancti Evangélii secúndum Lucam.%
}{%
✠ Suite du Saint Evangile selon saint Luc.%
}

\versio{%
℟. Glória tibi, Dómine.%
}{%
℟. Gloire à vous, Seigneur.%
}

\versio{%
In illo témpore : Missus est Angelus Gábriël a Deo in civitátem Galilǽæ, cui nomen Názareth, ad Vírginem desponsátam viro, cui nomen erat Ioseph, de domo David, et nomen Vírginis María. Et ingréssus Angelus ad eam, dixit : Ave, grátia plena ; Dóminus tecum : benedícta tu in muliéribus.%
}{%
En ce temps-là, l'Ange Gabriel fut envoyé de Dieu dans une ville de Galilée, appelée Nazareth, auprès d'une vierge fiancée à un homme de la maison de David, nommé Joseph ; et le nom de la vierge était Marie. L'ange, étant entré auprès d'elle, lui dit : Je vous salue, pleine de grâce ; le Seigneur est avec vous, vous êtes bénie entre les femmes.%
}

\subsection*{Offertoire}

\vulgo{Je vous salue, Marie, pleine de grâce : le Seigneur est avec vous : vous êtes bénie entre les femmes, Alléluia.}

\cantus{Offertoire}{AveMaria}{Off.}{8.}

\subsection*{Secrète}

\versio{%
Salutárem hóstiam, quam in commemoratióne immaculátæ Conceptiónis beátæ Vírginis Maríæ tibi, Dómine, offérimus, súscipe et præsta : ut, sicut illam tua grátia præveniénte ab omni labe immúnem profitémur ; ita eius intercessióne a culpis ómnibus liberémur. Per Dóminum nostrum Iesum Christum Fílium tuum, qui tecum vivit et regnat in unitáte Spíritus Sancti Deus, per ómnia sǽcula sæculórum.%
}{%
Recevez, Seigneur, l'hostie salutaire que nous vous offrons en la commémoraison de l'immaculée Conception de la bienheureuse Vierge Marie, et accordez-nous que, de même que nous confessons qu'elle a été exempte de toute tache par votre grâce qui l'a prévenue, nous soyons ainsi, par son intercession, délivrés de toutes nos fautes. Par Notre Seigneur Jésus-Christ votre Fils, qui, étant Dieu, vit et règne avec vous en l'unité du Saint-Esprit, dans tous les siècles des siècles.%
}

\subsection*{Préface}

\versio{%
Vere dignum et iustum est,
æquum et salutáre,
nos tibi semper et ubíque grátias ágere :
Dómine, sancte Pater, omnípotens ætérne Deus :
}{%
Il est vraiment juste et nécessaire,
c'est notre devoir et c'est notre salut,
de vous rendre grâces toujours et partout,
Seigneur, Père saint, Dieu éternel et tout-puissant :%
}

\versio{%
Et te in Conceptióne immaculáta beátæ Maríæ semper Vírginis
collaudáre, benedícere et prædicáre.
Quæ et Unigénitum tuum Sancti Spíritus obumbratióne concépit :
et, virginitátis glória permanénte,
lumen ætérnum mundo effúdit,
Iesum Christum, Dóminum nostrum.

Per quem maiestátem tuam laudant Angeli,
adórant Dominatiónes,
tremunt Potestátes.
Cæli cælorúmque Virtútes ac beáta Séraphim
sócia exsultatióne concélebrant.
Cum quibus et nostras voces ut admítti iúbeas, deprecámur,
súpplici confessióne dicentes.%
}{%
Et, en l'immaculée Conception de la bienheureuse Marie toujours Vierge
de vous louer, de vous bénir et de vous proclamer.
C'est elle qui a conçu votre Fils unique par l'opération du Saint-Esprit :
et qui, sans rien perdre de la gloire de sa virginité,
a mis au monde la lumière éternelle,
Jésus-Christ, Notre-Seigneur.

Par Lui les Anges louent votre majesté,
les Dominations vous adorent,
les Puissances se prosternent en tremblant.
Les Cieux, les Vertus des cieux et les bienheureux Séraphins
la célèbrent, unis dans une même allégresse.
À leurs chants, nous vous prions, laissez se joindre aussi nos voix
pour proclamer dans une humble louange.%
}

\subsection*{Communion}

\vulgo{Des choses glorieuses ont été dites de vous, Marie, parce que le Tout-Puissant a fait en vous de grandes choses.}

\cantus{Communion}{Gloriosa}{Comm.}{8.}

\subsection*{Postcommunion}

\oratio{%
Sacraménta quæ súmpsimus, Dómine, Deus noster : illíus in nobis culpæ vúlnera réparent ; a qua immaculátam beátæ Maríæ Conceptiónem singuláriter præservásti. Per Dóminum nostrum Iesum Christum Fílium tuum, qui tecum vivit et regnat in unitáte Spíritus Sancti Deus, per ómnia sǽcula sæculórum.%
}{%
{\small Seigneur, notre Dieu, que ce sacrement que nous avons reçu, répare en nous les blessures de cette faute dont, par un privilège tout spécial, vous avez préservé la bienheureuse Marie dans sa Conception immaculée. Par Notre Seigneur Jésus-Christ votre Fils, qui, étant Dieu, vit et règne avec vous en l'unité du Saint-Esprit, dans tous les siècles des siècles.}%
}


%%%%%%%%%%%%%%%%%%%%% Consécration au Cœur immaculé %%%%%%%%%%%%%%%%%%%%%%%%%%%%

\bigskip\needspace{.4\paperheight}
{\centering%
{\LARGE\bfseries%
Acte de consécration\\
de la Fraternité sacerdotale Saint-Pie X\\
à la Très Sainte Vierge Marie\\
et à son Cœur Douloureux et Immaculé\par}
\smallskip\par
}
\addcontentsline{toc}{chapter}{Consécration au Cœur immaculé}
\medskip

\begin{multicols}{2}
Nous recourons à Vous, ô Immaculée Mère de Dieu, en cette heure tragique de l’humanité et, plus encore, au sein de cette tempête sans précédent qui ébranle l’Église de fond en comble. Vous qui autrefois, debout au pied de la croix, avez compati si intimement aux souffrances de votre divin Fils, comment ne compatiriez-vous pas aujourd’hui à la Passion de l’Église son Corps Mystique !

Tandis qu’au dehors le communisme répand partout ses erreurs jusqu’à en infecter l’Église, c’est au sein même de celle-ci que le venin du faux œcuménisme empoisonne d’innombrables âmes, égarant les unes et maintenant les autres hors de l’unité de la vraie foi et de l’unique Arche du Salut.

Au milieu de tant de ruines et de trahisons, il a plu à Dieu selon l’antique exemple de susciter notre Fraternité Sacerdotale comme une petite armée de rebâtisseurs. Mais consciente de sa faiblesse, elle se tourne aujourd’hui vers Vous, Vierge Puissante, Secours des Chrétiens. Devant l’ampleur de notre mission, nous défiant de nos propres forces, nous voulons nous placer sous votre maternelle et puissante protection, ô Vierge redoutable comme une armée rangée en bataille, Vous qui avez reçu dès le commencement la promesse d’écraser la tête du serpent. Au milieu des dangers qui nous menacent, nous supplions Dieu de daigner sceller par Vous notre vocation à servir son Église, ô Arche d’Alliance !

C’est pourquoi, ô Vierge Immaculée, nous nous prosternons aujourd’hui aux pieds de votre trône de grâce, et désireux d’accroître votre louange et votre gloire ; afin de joindre notre petite part à l’amour filial du Christ votre Fils envers Vous, notre très douce Mère, nous Vous consacrons, au titre très spécial de votre Cœur Douloureux et Immaculé, et d’une manière irrévocable, notre Fraternité Sacerdotale Saint-Pie X, ses prêtres, ses séminaristes et ses Frères, ses Sœurs, ses Oblates et son Tiers-Ordre, toute sa famille spirituelle.

Et afin que Vous soyez désormais la Souveraine de notre Fraternité, par un acte de donation perpétuelle entre vos mains, nous Vous offrons et remettons tous nos biens et nos maisons afin que Vous en soyez la vraie Propriétaire ; nous Vous livrons et consacrons nos corps et nos âmes, tout nous-mêmes, afin que Vous disposiez de nous à votre gré. Les âmes également qui nous sont confiées nous vous les livrons pour que vous les gardiez sous votre tutelle maternelle, nous Vous confions et abandonnons enfin notre apostolat, afin qu’il soit désormais votre apostolat, ô Reine des Apôtres !

Notre Fraternité est dès lors votre domaine ; tenez-la si fermement, ô Tour de David, qu’elle ne puisse jamais s’écarter du bon chemin. Gardez, ô Vierge Fidèle, chacun de ses membres inébranlablement attaché à elle. Gardez notre foi virginale, ô Vierge très pure, Vous qui avez reçu le pouvoir d’exterminer les hérésies dans le monde entier ! Conservez à l’Église, ô Pleine de Grâce, le Sacrifice de la Messe dans son rite romain antique et vénérable, porteur de grâce, et gardez nous fidèles à lui. Faites fleurir parmi nous, ô Reine de tous les saints, la sainteté sacerdotale, religieuse et familiale. Gardez notre Fraternité, ô Mère de la divine Grâce, comme un rameau fructueux et toujours vivant de la Sainte Église Catholique et Romaine. Obtenez-nous, ô Mère de l’Église, la grâce d’être un instrument toujours plus docile et plus apte entre les mains de Dieu pour le salut du plus grand nombre d’âmes possible. Et afin que nous puissions reconnaître que vous avez exaucé nos prières, ô Vierge Clémente, envoyez-nous beaucoup de ces ouvriers que le divin Maître de la moisson appelle à sa moisson. Accordez-nous enfin, ô Mère du Souverain Prêtre, la grâce de concourir à la restauration du sacerdoce catholique, et par là au rayonnement de l’âme sacerdotale du Christ, qui amènera finalement l’établissement de son Règne sur les individus, les familles et les États.

Forts de notre titre d’apôtres de Jésus et de Marie, nous Vous promettons, ô Reine des martyrs et des confesseurs, de travailler jusqu’à notre dernier souffle à la restauration de toutes choses dans le Christ, à l’accroissement de son Règne et au glorieux triomphe de votre Cœur douloureux et Immaculé, ô Marie ! Amen.
\end{multicols}


%%%%%%%%%%%%%%%%%%%%%%%%%%%%%%%%%%%%%%%%%%%%%%%%%%%%%%%%%%%%%%%%%%%%%%%%%%%%%%%%
%%%%%%%%%%%%%%%%%%%%%%%%%%%%%%%%%% Dimanche 26 %%%%%%%%%%%%%%%%%%%%%%%%%%%%%%%%%
%%%%%%%%%%%%%%%%%%%%%%%%%%%%%%%%%%%%%%%%%%%%%%%%%%%%%%%%%%%%%%%%%%%%%%%%%%%%%%%%


\part[Dimanche 26 octobre]{%
\bgimage{ChristRoi}\\
Dimanche 26 octobre\\\ \\ {\large Fête du Christ-Roi\\\ \\ \emph{Il faut qu'il règne.}}}

\vspace*{\stretch{1}}
\section*{Programme de la journée}

\begin{tabularx}{\textwidth-\parindent}{l!{:}X}
9h00	& Chapelet.
	  \lieu{Basilique Saint-Pie X}\\
9h30	& Messe du Christ-Roi, célébrée par son Excellence Mgr Bernard \textsc{Fellay}, suivie d'une procession solennelle à la Grotte, et \textbf{consécration de la Fraternité Saint-Pie X au Christ-Roi, Prince de la paix, Maître des nations}.
	  \lieu{Basilique Saint-Pie X}\linebreak
	  Ministres supérieurs :\hspace*{\stretch{1}}\linebreak
	  − Prêtre-assistant : M. l'abbé Christian \textsc{Bouchacourt}, supérieur du district de France ;\hspace*{\stretch{1}}\linebreak
	  − Diacres-assistants : M. l'abbé Davide \textsc{Paglariani}, supérieur du séminaire de La Reja (Argentine), et M. l'abbé Vicente \textsc{Griego}, supérieur du séminaire d'Holy-Cross (Australie).\hspace*{\stretch{1}}\linebreak
	  Service liturgique assuré par le séminaire international Saint-Pie X, d'Écône (Suisse).\\
15h30	& II\iemes\ Vêpres, procession du Saint-Sacrement dans les rues de Lourdes, et bénédiction des malades.
	  \lieu{Basilique Saint-Pie X}\\
18h00	& Conférence\footnote{Les conférences seront données en français, italien, espagnol, anglais et allemand.} : \emph{Les combats de saint Pie X pour la royauté du Christ}.
	  \lieu{Église Sainte-Bernadette}\\
20h30	& Exposition du Saint-Sacrement − Nuit d'adoration jusqu'a 8h le lendemain matin.
	  \lieu{Basilique Saint-Pie X}
\end{tabularx}

\vfill
{\noindent\footnotesize Une ancienne tradition du Pays-Basque, exprimant le règne social de Notre Seigneur Jésus-Christ, autorise une escorte militaire auprès du Saint-Sacrement. La garde royale de Navarre est constituée des élèves et paroissiens du prieuré-école Saint-Michel Garicoïtz de Domezain.}

\vspace{\stretch{2}}

%%%%%%%%%%%%%%%%%%%%%%%%%%%%%% Messe du Christ-Roi %%%%%%%%%%%%%%%%%%%%%%%%%%%%%

\addchap{Messe du Christ-Roi}

\subsection*{Introït}

\vulgo{Il est digne, l'Agneau qui a été immolé, de recevoir la puissance, la divinité, la sagesse, la force, l'honneur. À Lui la gloire et le pouvoir dans les siècles des siècles.
℣. Ô Dieu, donnez au Roi votre jugement, et au Fils du Roi votre justice.}

\cantus{Introit}{DignusEstAgnus}{Intr.}{3.}

{\centering\emph{Kyriale II − Credo III}\par}

\vspace{-1\baselineskip}

\subsection*{Collecte}

\oratio{%
Omnípotens sempitérne Deus, qui in dilécto Fílio tuo, universórum Rege, ómnia instauráre voluísti : concéde propítius ; ut cunctæ famíliæ géntium, peccáti vúlnere disgregátæ, eius suavissímo subdántur império : Qui tecum vivit et regnat in unitáte Spíritus Sancti Deus, per ómnia sǽcula sæculórum.
}{%
Dieu tout-puissant et éternel, qui avez voulu restaurer tout dans la personne de votre Fils bien-aimé, le Roi de l'univers : accordez dans votre bonté, que toutes les familles des nations, qui vivent en désaccord à cause de la blessure du péché, se soumettent à son très doux pouvoir. Lui qui, étant Dieu, vit et règne avec vous en l'unité du Saint-Esprit, dans tous les siècles des siècles.
}

\vspace{-1\baselineskip}

\subsection*{Épître}

\versio{%
Léctio Epístolæ beáti Pauli Apóstoli ad Colossénses.

Fratres : Grátias ágimus Deo Patri, qui dignos nos fecit in partem sortis sanctórum in lúmine : qui eripuit nos de potestáte tenebrárum, et tránstulit in regnum Fílii dilectiónis suæ, in quo habémus redemptiónem per sánguinem eius, remissiónem peccatórum : qui est imágo Dei invisíbilis, primogénitus omnis creatúra : quóniam in ipso cóndita sunt univérsa in cælis et in terra, visibília et invisibília, sive Throni, sive Dominatiónes, sive Principátus, sive Potestátes : ómnia per ipsum, et in ipso creáta sunt : et ipse est ante omnes, et ómnia in ipso constant. Et ipse est caput córporis Ecclésiæ, qui est princípium, primogénitus ex mórtuis : ut sit in ómnibus ipse primátum tenens ; quia in ipso complácuit omnem plenitúdinem inhabitáre ; et per eum reconciliáre ómnia in ipsum, pacíficans per sánguinem crucis eius, sive quæ in terris, sive quæ in cælis sunt, in Christo Iesu, Dómino nostro.%
}{{\small%
Lecture de l'Épître de Saint Apôtre Paul aux Colossiens.

Mes Frères : Rendons grâces à Dieu le Père, qui nous a rendus dignes d'avoir part à l'héritage des saints dans la lumière, qui nous a arrachés à la puissance des ténèbres, et nous a fait passer dans le royaume de son Fils bien-aimé, en qui nous avons la rédemption, par son sang et la rémission des péchés ; qui est l'image du Dieu invisible, le premier-né de toute créature. Car c'est par lui que toutes choses ont été créées dans les cieux et sur la terre, les visibles et les invisibles, soit Trônes, soit Dominations, soit Principautés, soit Puissances : tout a été créé par lui, et en lui ; et lui-même est avant tous, et tout subsiste en lui. Et lui-même est le chef du corps de l'Église : il est le principe, le premier-né d'entre les morts, afin qu'en toutes choses il garde la primauté ; parce qu'il a plu au Père que toute plénitude habitât en lui ; et de se réconcilier par lui toutes choses, pacifiant par le sang de sa croix, soit ce qui est sur la terre, soit ce qui est dans les cieux, en Jésus-Christ Notre-Seigneur.%
}}

\subsection*{Graduel}

\vulgo{Il dominera de la mer à la mer, et depuis le fleuve jusqu'aux extrémités de la terre. ℣. Et tous les rois de la terre l'adoreront, toutes les nations lui seront assujetties.}

\cantus{Graduel}{Dominabitur}{Grad.}{5.}

\subsection*{Alléluia}

\vulgo{Alléluia, alléluia. ℣. Sa puissance est une puissance éternelle qui ne sera pas emportée, et son règne est un règne qui ne sera point bouleversé. Alléluia.}

\cantus{Alleluia}{PotestasEius}{}{1.}

\vspace{0\baselineskip minus 1\baselineskip}
\subsection*{Évangile}

\versio{%
✠ Sequéntia sancti Evangélii secúndum Ioánnem.%
}{%
✠ Suite du Saint Evangile selon saint Jean.%
}

\versio{%
℟. Glória tibi, Dómine.%
}{%
℟. Gloire à vous, Seigneur.%
}

\versio{%
In illo témpore : Dixit Pilátus ad Iesum : Tu es Rex Iudæórum ? Respóndit Iesus : A temetípso hoc dicis, an alii dixérunt tibi de me ? Respóndit Pilátus : Numquid ego Iudǽus sum ? Gens tua et pontífices tradidérunt te mihi : quid fecísti ? Respóndit Iesus : Regnum meum non est de hoc mundo. Si ex hoc mundo esset regnum meum, minístri mei útique decertárent, ut non tráderer Iudǽis : nunc autem regnum meum non est hinc. Dixit ítaque ei Pilátus : Ergo Rex es tu ? Respóndit Iesus : Tu dicis, quia Rex sum ego. Ego in hoc natus sum et ad hoc veni in mundum, ut testimónium perhíbeam veritáti : omnis, qui est ex veritáte, audit vocem meam.%
}{%
En ce temps-là : Pilate dit à Jésus : Es-tu le roi des Juifs ? Jésus répondit : Dis-tu cela de toi-même, ou d'autres te l'ont-ils dit de moi ? Pilate répondit : Est-ce que je suis Juif, moi ? Ta nation et les princes des prêtres t'ont livré à moi ; qu'as-tu fait ? Jésus répondit : Mon royaume n'est pas de ce monde. Si mon royaume était de ce monde, mes serviteurs auraient combattu, pour que je ne fusse pas livré aux Juifs ; mais mon royaume n'est point d'ici. Pilate lui dit alors : Tu es donc roi ? Jésus répondit : Tu le dis, je suis roi. Voici pourquoi je suis né, et pourquoi je suis venu dans le monde : pour rendre témoignage à la vérité. Quiconque est de la vérité, écoute ma voix.%
}

\subsection*{Offertoire}

\vulgo{Demande-moi, et je te donnerai les nations pour ton héritage, et pour ton domaine les extrémités de la terre.}\pagebreak[3]

\cantus{Offertoire}{PostulaAMe}{Off.}{4.}

\needspace{4\baselineskip}
\subsection*{Secrète}

\versio{%
Hóstiam tibi, Dómine, humánæ reconciliatiónis offérimus : præsta, quǽsumus ; ut, quem sacrifíciis præséntibus immolámus, ipse cunctis géntibus unitátis et pacis dona concédat, Iesus Christus Fílius tuus, Dóminus noster : Qui tecum vivit et regnat in unitáte Spíritus Sancti Deus, per ómnia sǽcula sæculórum.%
}{%
Nous vous offrons, Seigneur, le sacrifice de la réconciliation de l'homme : faites, nous vous prions, que Celui que nous immolons dans ce sacrifice, accorde Lui-même à toutes les nations les dons d'unité et de paix, Jésus-Christ votre Fils, notre Seigneur, qui étant Dieu vit et règne avec vous en l'unité du Saint-Esprit, dans tous les siècles des siècles.%
}

\subsection*{Préface}

\versio{%
Vere dignum et iustum est, æquum et salutáre,
nos tibi semper et ubíque grátias ágere :
Dómine, sancte Pater, omnípotens ætérne Deus :

Qui unigénitum Fílium tuum, Dóminum nostrum Iesum Christum,
Sacerdótem ætérnum et universórum Regem,
óleo exsultatiónis unxísti :
ut seípsum in ara crucis
hóstiam immaculátam et pacíficam ófferens,
redemptiónis humánæ sacraménta perágeret :
et suo subiéctis império ómnibus creatúris,
ætérnum et universále regnum, imménsæ tuæ tráderet Maiestáti.
Regnum veritátis et vitæ :
Regnum sanctitátis et grátiæ :
Regnum iustítiæ, amóris et pacis.

Et ídeo cum Angelís et Archángelis,
cum Thronis et Dominatiónibus,
cumque omni milítia cæléstis exércitus,
hymnum glóriæ tuæ cánimus, sine fine dicéntes.%
}{%
Il est vraiment juste et nécessaire,
c'est notre devoir et c'est notre salut,
de vous rendre grâces toujours et partout,
Seigneur, Père saint, Dieu éternel et tout-puissant :
Qui avez oint avec l'huile d'allégresse
votre Fils unique, Notre Seigneur Jésus-Christ,
Prêtre éternel et Roi de l'univers :
pour que s'immolant lui-même sur l'autel de la croix,
comme une victime sans tache et pacifique,
il accomplît le mystère sacré de la rédemption de l'homme :
et qu'après avoir soumis toutes les créatures à son pouvoir,
il procurât à votre immense Majesté un royaume éternel et universel,
un royaume de vérité et de vie,
un royaume de sainteté et de grâce,
un royaume de justice, d'amour et de paix.

C'est pourquoi avec les Anges et les Archanges,
avec les Trônes et les Dominations,
avec toute l'armée céleste,
nous chantons une hymne à votre gloire, disant sans cesse.%
}

\subsection*{Communion}

\vulgo{Le Seigneur siègera comme Roi éternellement : le Seigneur bénira son peuple dans la paix.}

\cantus{Communion}{SedebitDominus}{Comm.}{6.}

\needspace{6\baselineskip}
\subsection*{Postcommunion}

\oratio{%
Immortalitátis alimóniam consecúti, quǽsumus, Dómine : ut, qui sub Christi Regis vexíllis militáre gloriámur, cum ipso, in cælésti sede, iúgiter regnáre póssimus : Qui tecum vivit et regnat in unitáte Spíritus Sancti Deus, per ómnia sǽcula sæculórum.%
}{{\small%
Ayant reçu l'aliment de l'immortalité, nous vous prions, Seigneur : puissions-nous, qui nous glorifions de combattre sous l'étendard du Christ, régner toujours avec Lui dans le céleste séjour. Lui qui, étant Dieu, vit et règne avec vous en l'unité du Saint-Esprit, dans tous les siècles des siècles.%
}}


%%%%%%%%%%%%%%%%%%%%%%%%%% Consécration au Christ-Roi %%%%%%%%%%%%%%%%%%%%%%%%%%

\bigskip\needspace{.4\paperheight}
{\LARGE\bfseries\centering%
Consécration de la Fraternité Sacerdotale \mbox{Saint-Pie X}
au Christ-Roi, Prince de la Paix et Maître des Nations\par}
\addcontentsline{toc}{chapter}{Consécration au Christ-Roi}
\medskip

\begin{multicols}{2}
Ô Jésus, Seigneur et Maître de toutes choses, nous nous prosternons à vos pieds pour vous adorer et vous reconnaître pour notre Chef et notre Roi. À vous toutes les nations sont soumises, car vous êtes seul le vrai Roi, la vraie Paix et la vraie Lumière. Nous n'adorons que vous seul, vous êtes notre Soutien, notre Espérance et notre Salut, ô grand Dieu du ciel et de la terre.

Nous consacrons donc à votre Cœur de Roi notre Fraternité Sacerdotale Saint-Pie X, chacun de ses membres et toutes ses œuvres. Prenez cette Fraternité, qu'elle soit tout à vous. Embrasez les cœurs de tous ses membres des flammes de votre charité, et consumez-les dans votre amour. Nous vous confions aussi toutes nos peines et nos besoins. Disposez de chacun de nous selon votre bon plaisir, nous remettons tout entre vos mains. Nous n'attendons de secours que de vous. 

Ô Christ-Roi, nous vous rendons tout honneur et toute gloire. Nous voulons vous honorer jusqu'à notre dernier soupir, en travaillant inlassablement à votre règne, pour que votre volonté soit faite sur la terre comme au ciel.

Soyez vraiment, au milieu de notre Fraternité qui est vôtre, le Prince de la paix ; écrasez de votre sceptre le démon de la révolte et de la division ; faites resplendir en ses ministres l’amour du sacerdoce catholique dans toute sa pureté doctrinale et sa charité missionnaire ; que par leur ministère ils affermissent votre trône royal et rendent hommage à votre Loi sainte, en laquelle s'unissent la justice et la miséricorde.

Et que par vous la Fraternité Sacerdotale Saint-Pie X, fidèle à sa vocation, soutenue dans son action par la puissance de la prière, par la concorde dans la charité, par une ferme et indéfectible vigilance, exalte dans le monde le triomphe et le règne de votre nom : Christ-Roi, Prince de la paix et Maître des nations.

Ainsi soit-il.

\noindent\rule{.5\linewidth}{1sp}

{\footnotesize
\noindent Sources :
\begin{itemize}
\item Consécration officielle du monastère des Bénédictines du Saint Sacrement le 28 mars 1927 ;
\item Prière au Christ-Roi, publiée dans ce même monastère le 1er juillet 1927 ;
\item Prière à Marie Reine de France, indulgenciée par Pie XII le 15 mars 1948.
\end{itemize}
}
\end{multicols}


%%%%%%%%%%%%%%%%%%%%%%%%%%%%% Vêpres du Christ-Roi %%%%%%%%%%%%%%%%%%%%%%%%%%%%%

\addchap{Vêpres du Christ-Roi}

\cantus{Verset}{DeusInAdiutorium_solemnis}{}{℣.}

\cantus{Antienne}{Pacificus-ps109}{Ant.}{8.G}

%\cantus{Psaume}{Intonation-109-8G}{}{8.G}

\psalmus[tonus=8G,primus=2,numerus=2]{109}
\gloria[tonus=8G]

\medskip
\cantus{Antienne}{RegnumEius-ps110}{Ant.}{8.c}

%\cantus{Psaume}{Intonation-110-8c}{}{8.c}

\psalmus[tonus=8c,primus=2,numerus=2]{110}
\gloria[tonus=8c]

\medskip
\cantus{Antienne}{EcceVirOriens-ps111}{Ant.}{7.a}

%\cantus{Psaume}{Intonation-111-7a}{}{7.a}

\psalmus[tonus=7a,primus=2,numerus=2]{111}
\gloria[tonus=7a]

\medskip
\cantus{Antienne}{DominusIudexNoster-ps112}{Ant.}{3.a}

%\cantus{Psaume}{Intonation-112-3a}{}{3.a}

\psalmus[tonus=3a,primus=2,numerus=2]{112}
\gloria[tonus=3a]

\medskip
\cantus{Antienne}{EcceDediTe-ps116}{Ant.}{8.G}

%\cantus{Psaume}{Intonation-116-8G}{}{8.G}

\psalmus[tonus=8G,primus=2,numerus=2]{116}
\gloria[tonus=8G]


\vspace{1\baselineskip}
\subsection*{Capitule}

\versio{%
Fratres : Grátias ágimus Deo Patri, qui dignos nos fecit in partem sortis sanctórum in lúmine : † qui erípuit nos de potestáte tenebrárum, * et tránstulit in regnum Fílii dilectiónis suæ.%
}{%
Mes frères : nous rendons grâces à Dieu le Père, qui nous a rendus capables d'avoir part à l'héritage des saints dans la lumière, en nous délivrant de la puissance des ténèbres, pour nous transporter dans le royaume de son Fils bien-aimé.%
}

\versio{%
℟. Deo grátias.%
}{%
℟. Rendons grâces à Dieu.%
}

\subsection*{Hymne}

\cantus{Hymne}{TeSaeculorumPrincipem}{Hymn.}{1.}


\subsection*{Magníficat}

\cantus{Antienne}{HabetInVestimento}{Ant.}{7.a}

\cantus{Psaume}{Intonation-Magnificat-7a}{}{7.a}

\canticum[tonus=7a,primus=2,numerus=2]{Magnificat}
\gloria[tonus=7a]


\subsection*{Oraison}

\versio{%
℣. Dóminus vobíscum.%
}{%
℣. Le Seigneur soit avec vous.%
}

\versio{%
℟. Et cum spíritu tuo.%
}{%
℟. Et avec votre esprit.%
}

\oratio{%
Omnípotens sempitérne Deus, qui in dilécto Fílio tuo, universórum Rege, ómnia instauráre voluísti : † concéde propítius ; ut cunctæ famíliæ Géntium, peccáti vúlnere disgregátæ, * eius suavíssimo subdántur império : Qui tecum vivit et regnat in unitáte Spíritus Sancti Deus, per ómnia sǽcula sæculórum.%
}{%
Dieu tout-puissant et éternel, qui avez voulu restaurer toutes choses en votre Fils bien-aimé, roi de l'univers, accordez avec bonté que toutes les familles des nations, déchirées par la blessure du péché, soient soumises à son très doux empire ; lui qui, étant Dieu, vit et règne avec vous en l'unité du Saint-Esprit, dans tous les siècles des siècles.%
}

\subsection*{Conclusion}

\versio{%
℣. Dóminus vobíscum.%
}{%
℣. Le Seigneur soit avec vous.%
}

\versio{%
℟. Et cum spíritu tuo.%
}{%
℟. Et avec votre esprit.%
}

\medskip
\cantus{Verset}{BenedicamusDomino-IIVepres-IClasse-AdLib}{℣.}{}


%%%%%%%%%%%%%%%%%%%%%%%%%%%%%%%%%%%%%%%%%%%%%%%%%%%%%%%%%%%%%%%%%%%%%%%%%%%%%%%%
%%%%%%%%%%%%%%%%%%%%%%%%%%%%%%%%%%% Lundi 27 %%%%%%%%%%%%%%%%%%%%%%%%%%%%%%%%%%%
%%%%%%%%%%%%%%%%%%%%%%%%%%%%%%%%%%%%%%%%%%%%%%%%%%%%%%%%%%%%%%%%%%%%%%%%%%%%%%%%

\part[Lundi 27 octobre]{%
\bgimage{StPieX}\\
Lundi 27 octobre\\\ \\ {\large En l'honneur de saint Pie X\\\ \\ \emph{Tout instaurer dans le Christ.}}}

\vspace*{\stretch{1}}
\section*{Programme de la journée}

\begin{tabularx}{\textwidth-\parindent}{l!{:}X}
9h00	& Chapelet.
	\lieu{Basilique Saint-Pie X}\\
9h30	& Messe de saint Pie X, célébrée par son Excellence Mgr Bernard \textsc{Tissier de Mallerais}.
	  \lieu{Basilique Saint-Pie X}\linebreak
	  Ministres supérieurs :\hspace*{\stretch{1}}\linebreak
	  − Prêtre-assistant : M. l'abbé Franz \textsc{Schmidberger}, supérieur du séminaire de Zaitzkofen ;\hspace*{\stretch{1}}\linebreak
	  − Diacres-assistants : MM. les abbés Firmin \textsc{Udressy}, supérieur du district d'Allemagne, et Stefan \textsc{Frey}, supérieur du district d'Autriche.\hspace*{\stretch{1}}\linebreak
	  Service liturgique assuré par le séminaire international du Cœur de Jésus, de Zaitzkofen (Allemagne).\\
11h15	& Chapelet médité à la Grotte, puis \textbf{supplique à saint Pie X} pour préserver toutes les œuvres de la Tradition.\\
12h15	& Clôture du pèlerinage.
\end{tabularx}

\vspace{\stretch{2}}

%%%%%%%%%%%%%%%%%%%%%%%%%%%%%%%% Messe de St Pie X %%%%%%%%%%%%%%%%%%%%%%%%%%%%%

\addchap{Messe de S. Pie X}

\subsection*{Introït}

\vulgo{J'ai élevé celui que j'ai choisi du milieu du peuple, je l'ai oint de mon huile sainte, pour que ma main l'assiste toujours, et mon bras le fortifie. ℣. Je chanterai éternellement les miséricordes du Seigneur ; de génération en génération ma bouche annoncera votre fidélité.}

\enlargethispage{2\baselineskip}
\cantus{Introit}{ExtuliElectum}{Intr.}{3.}

\vspace{\stretch{1}}
{\centering\emph{Kyriale VIII}\par}
\vspace*{\stretch{1}}\pagebreak

\subsection*{Collecte}

\vspace{-.2\baselineskip plus .2\baselineskip}
\oratio{%
Deus, qui, ad tuéndam cathólicam fidem et univérsa in Christo instauránda, sanctum Pium Summum Pontíficem cælésti sapiéntia et apostólica fortitúdine replevísti : concéde propítius ; ut, eius institúta et exémpla sectántes, prǽmia consequámur ætérna. Per eúndem Dóminum nostrum Iesum Christum Fílium tuum, qui tecum vivit et regnat in unitáte Spíritus Sancti Deus, per ómnia sǽcula sæculórum.%
}{%
Ô Dieu, qui pour protéger la foi catholique et instaurer toute chose dans le Christ, avez rempli le Souverain Pontife saint Pie X de sagesse céleste et de force apostolique : accordez-nous favorablement de suivre son enseignement et ses exemples afin de parvenir aux biens éternels. Par Notre Seigneur Jésus-Christ votre Fils, qui, étant Dieu, vit et règne avec vous en l'unité du Saint-Esprit, dans tous les siècles des siècles.%
}

\subsection*{Épître}

\vspace{-.2\baselineskip plus .2\baselineskip}
\versio{%
Lectio Epístolæ beáti Pauli Apóstoli ad Thessalonicénses.

Fratres : Fidúciam habúimus in Dómino nostro, loqui ad vos Evangélium Dei in multa sollicitúdine. Exhortátio enim nostra non de erróre, neque de immundítia, neque in dolo ; sed sicut probáti sumus a Deo, ut crederétur nobis Evangélium : ita lóquimur, non quasi homínibus placéntes, sed Deo, qui probat corda nostra. Neque enim aliquándo fúimus in sermóne adulatiónis, sicut scitis : neque in occasióne avarítiæ : Deus testis est : nec quæréntes ab homínibus glóriam, neque a vobis, neque ab áliis. Cum possémus vobis óneri esse, ut Christi Apóstoli ; sed facti sumus párvuli in médio vestrum, tamquam si nutrix fóveat fílios suos. Ita desiderántes vos, cúpide volebámus trádere vobis non solum Evangélium Dei, sed etiam ánimas nostras, quóniam caríssimi nobis facti estis.%
}{%
{\small Lecture de la première Épître de saint Paul aux Thessaloniciens.

Mes frères : Si, pour vous annoncer l'Évangile en dépit de tant de difficultés, nous avons montré une telle assurance, c'est en Dieu que nous l'avons trouvée. Notre prédication ne procède ni de l'erreur, ni d'intentions impures ; elle n'use pas de diplomatie. Mais puisque Dieu nous a jugé digne de nous confier son Évangile, nous ne parlons pas pour plaire aux hommes, mais à Dieu qui juge notre cœur. De fait, à aucun moment, nous n'avons employé des paroles de flatterie, vous le savez bien. Jamais nous n'avons cherché de profits personnels, Dieu en est témoin ; nous n'avons pas ambitionné une célébrité parmi les hommes, ni chez vous, ni ailleurs. Comme apôtres du Christ, nous aurions pu cependant rester à votre charge ; mais nous nous sommes comportés parmi vous avec une simplicité d'enfants. Et comme une mère entoure de tendresse les enfants qu'elle nourrit, dans notre affection pour vous, nous désirons vivement vous donner non seulement l'Évangile de Dieu, mais encore notre vie. Car vous êtes devenus très chers à notre cœur.}%
}

\subsection*{Graduel}

\vulgo{J'ai publié la justice dans la grande assemblée : je n'ai pas fermé mes lèvres, Seigneur, vous le savez.
℣. Je n'ai pas caché votre justice dans mon cœur ; j'ai proclamé votre vérité et votre salut.}

\cantus{Graduel}{Annuntiavi}{Grad.}{5.}

\subsection*{Alléluia}

\vulgo{Alléluia, Alléluia. ℣. Vous avez préparé devant moi une table, vous avez oint ma tête d'huile, et que mon calice est admirable !}

\cantus{Alleluia}{ParasMihi}{}{8.}

\vspace{-1\baselineskip plus 0\baselineskip minus 1\baselineskip}
\subsection*{Évangile}

\vspace{-.2\baselineskip plus .2\baselineskip}
\versio{%
✠ Sequéntia sancti Evangélii secúndum Ioánnem.%
}{%
✠ Suite du Saint Evangile selon saint Jean.%
}

\versio{%
℟. Glória tibi, Dómine.%
}{%
℟. Gloire à vous, Seigneur.%
}

\versio{%
In illo témpore : Dixit Iesus Simóni Petro : Simon Ioánnis, díligis me plus his ? Dicit ei : Etiam, Dómine, tu scis, quia amo te. Dicit ei : Pasce agnos meos. Dicit ei íterum : Simon Ioánnis, díligis me ? Ait illi : Etiam, Dómine, tu scis, quia amo te. Dicit ei : Pasce agnos meos. Dicit ei tértio : Simon Ioánnis, amas me ? Contristátus est Petrus, quia dixit ei tértio, Amas me ? et dixit ei : Dómine, tu ómnia nosti : tu scis, quia amo te. Dixit ei : Pasce oves meas.%
}{%
En ce temps-là, Jésus dit à Simon-Pierre : Simon, fils de Jean, m'aimes-tu plus que ceux-ci ? Il lui répondit : Oui, Seigneur, vous savez que je vous aime. Jésus lui dit : Pais mes agneaux. Il lui dit de nouveau : Simon, fils de Jean, m'aimes-tu ? Pierre lui répondit : Oui, Seigneur, vous savez que je vous aime. Jésus lui dit : Pais mes agneaux. Il lui dit pour la troisième fois : Simon, fils de Jean, m'aimes-tu ? Pierre fut attristé de ce qu'il lui avait dit pour la troisième fois : M'aimes-tu ? et il lui répondit : Seigneur, vous savez toutes choses ; vous savez que je vous aime. Jésus lui dit : Pais mes brebis.%
}

\subsection*{Offertoire}

\vulgo{Venez, mes fils, écoutez-moi : je vous enseignerai la crainte du Seigneur.}

\cantus{Offertoire}{Venite}{Off.}{4.}

\subsection*{Secrète}

\versio{%
Oblatiónibus nostris, quǽsumus, Dómine, benígne suscéptis, da nobis, ut hæc divína mystéria, sancto Pio Summo Pontífice intercedénte, sincéris tractémus obséquiis et fidéli mente sumámus. Per Dóminum nostrum Iesum Christum Fílium tuum, qui tecum vivit et regnat in unitáte Spíritus Sancti Deus, per ómnia sǽcula sæculórum.%
}{%
Nous vous en prions, Seigneur, acceptez avec bienveillance nos offrandes et donnez-nous, par l'intercession du Souverain Pontife saint Pie X, de célébrer ces divins mystères avec une piété sincère et d'y communier avec foi. Par Notre Seigneur Jésus-Christ votre Fils, qui, étant Dieu, vit et règne avec vous en l'unité du Saint-Esprit, dans tous les siècles des siècles.%
}

\subsection*{Préface}

\versio{%
Vere dignum et iustum est,
æquum et salutáre,
nos tibi semper et ubíque grátias ágere,
Dómine, sancte Pater, omnípotens, ætérne Deus :%
}{%
Il est vraiment juste et nécessaire,
c'est notre devoir et c'est notre salut,
de vous rendre grâces toujours et partout,
Seigneur, Père saint, Dieu éternel et tout-puissant :%
}

\versio{%
Qui glorificáris in concílio Sanctórum,
et eórum coronándo mérita,
corónas dona tua.
Qui nobis in eórum præbes,
et conversatióne exémplum,
et communióne consórtium,
et intercessióne subsídium ;
ut tantam habéntes impósitam nubes téstium,
per patiéntiam currámus
ad propósitum nobis certámen,
et cum eis
percipiámus immarcescíbilem glóriæ corónam.%
}{%
Vous trouvez votre gloire dans l'assemblée des Saints
et, en couronnant leurs mérites,
vous couronnez vos propres dons.
En eux, vous avez voulu que nous trouvions
une vie qui nous serve d'exemple,
une communion qui nous donne une famille,
une prière qui nous soit un secours ;
afin qu'environnés d'une telle nuée de témoins,
nous courrions sans défaillance
au combat qui nous est proposé
et recevions avec eux
la couronne impérissable de la gloire.%
}

\versio{%
Per Iesum Christum Dóminum nostrum,
cuius sánguine ministrátur nobis intróitus in ætérnum regnum.
Per quem Maiestátem tuam treméntes adórant Angeli,
et omnes Spirítuum cæléstium chori
sócia exsultatióne concélebrant.
Cum quibus et nostras voces ut admítti iúbeas deprecámur,
súpplici confessióne dicéntes.%
}{%
Par Jésus-Christ Notre-Seigneur,
dont le sang nous procure l'entrée au Royaume éternel.
Par lui les Anges adorent en tremblant votre Majesté,
et tous les chœurs des Esprits célestes
la célèbrent, unis dans une même allégresse.
À leurs chants, nous vous prions, laissez se joindre aussi nos voix
pour proclamer dans une humble louange.%
}

\subsection*{Communion}

\vulgo{Ma chair est vraiment une nourriture et mon sang une boisson. Celui qui mange ma chair et boit mon sang demeure en moi et moi en lui.}

\cantus{Communion}{CaroMea}{Comm.}{4}

\needspace{5\baselineskip}
\subsection*{Postcommunion}

\oratio{%
Mensæ cæléstis virtúte refécti, quǽsumus, Dómine Deus noster : ut, interveniénte sancto Pio Summo Pontífice ; fortes efficiámur in fide, et in tua simus caritáte concórdes. Per Dóminum nostrum Iesum Christum Fílium tuum, qui tecum vivit et regnat in unitáte Spíritus Sancti Deus, per ómnia sǽcula sæculórum.%
}{%
Réconfortés par ce repas céleste, nous vous prions, Seigneur, notre Dieu, que par l'intercession du Souverain Pontife saint Pie X, nous soyons rendus fermes dans la foi et que nous soyons uni dans votre charité. Par Notre Seigneur Jésus-Christ votre Fils, qui, étant Dieu, vit et règne avec vous en l'unité du Saint-Esprit, dans tous les siècles des siècles.%
}


%%%%%%%%%%%%%%%%%%%%%%%%%%% Supplique à Saint Pie X %%%%%%%%%%%%%%%%%%%%%%%%%%%%

\bigskip\needspace{.4\paperheight}
{\centering%
{\LARGE\bfseries%
Supplique à Saint Pie X, patron de notre Fraternité Sacerdotale\par}
\smallskip
à l'occasion du centenaire de son rappel à Dieu\par
}
\addcontentsline{toc}{chapter}{Supplique à Saint Pie X}
\medskip

\begin{multicols}{2}
Ô Bienheureux Pontife, fidèle Serviteur de votre Seigneur, humble et sûr disciple du Maître divin, Pasteur expérimenté du troupeau du Christ, tournez votre regard vers notre Fraternité Sacerdotale, placée sous votre patronage, et qui célèbre aujourd'hui le centième anniversaire de votre entrée au Ciel.

En ces temps difficiles, l’Épouse du Christ, autrefois confiée à vos soins, traverse à nouveau de graves difficultés. Ses fils sont menacés par d'innombrables périls dans leur âme et dans leur corps. Satan, comme un lion rugissant, rôde aux alentours, cherchant qui dévorer. Plus d'un devient sa victime, par attrait de l’esprit du monde et l’abandon du bon combat de la foi. Ils ont des yeux et ne voient pas ; ils ont des oreilles et n'entendent point. Ils ferment leur regard à la lumière de la vérité éternelle ; ils écoutent les voix des sirènes insinuant des messages trompeurs. Vous qui fûtes ici-bas un grand inspirateur et le guide du peuple chrétien, soyez notre aide et notre intercesseur, spécialement de tous ceux qui se lient à notre Fraternité : évêques, prêtres, séminaristes, frères, sœurs, oblates et tertiaires.

Vous dont le cœur se brisa quand vous vîtes le monde se précipiter dans une lutte sanglante, secourez l'humanité, venez en aide à la chrétienté, affermissez notre Fraternité sacerdotale. Obtenez de la miséricorde divine le don d'une paix durable, l’union des cœurs et une grande foi, qui seule peut ramener parmi les hommes et les nations la justice et la concorde voulues par Dieu.

Ô Saint Pie X, gloire du sacerdoce et honneur du peuple chrétien ; vous en qui l'humilité s’allia avec la grandeur, l'austérité avec la mansuétude, la piété simple avec la doctrine profonde ; vous, Pontife de l'Eucharistie et du catéchisme, de la foi intègre et de la charité apostolique ; tournez votre regard vers la Sainte Église et notre Fraternité ; obtenez-leur l'intégrité et la constance au milieu des difficultés et des persécutions de notre temps ; relevez cette pauvre humanité, aux douleurs de laquelle vous avez tellement pris part que cela finit par arrêter les battements de votre cœur.

Faites que la paix triomphe dans ce monde agité, cette paix qui doit être harmonieuse soumission de toutes les nations à l'autorité et à la loi de Jésus-Christ, l'Unique Roi de l'univers, par le Cœur douloureux et immaculé de la Très Sainte Vierge Marie et par votre intercession, ô glorieux saint Pie X. Nous vous le demandons, par le même Jésus-Christ Notre Seigneur, qui avec le Père et le Saint-Esprit vit et règne dans les siècles des siècles. 

Ainsi soit-il !

\noindent\rule{.4\linewidth}{1sp}

{\footnotesize
Sources :
\begin{itemize}
\item Discours du Pape Pie XII (Béatification de Pie X), le 3 juin 1951 ;
\item Discours du Pape Pie XII (Canonisation de Pie X) le, 29 mai 1954.
\end{itemize}
}
\end{multicols}


%%%%%%%%%%%%%%%%%%%%%%%%%%%%%%%%%%%%%%%%%%%%%%%%%%%%%%%%%%%%%%%%%%%%%%%%%%%%%%%%

\part{Cantiques}

\clearpage


\addchap{Chants au Christ-Roi}


\titre{Litanies du Sacré-Cœur}

\begin{litaniae}
\invocatio{Kýrie, eléison.}{Seigneur, ayez pitié de nous.}
\invocatio{Christe, eléison.}{Jésus-Christ, ayez pitié de nous.}
\invocatio{Kýrie, eléison.}{Seigneur, ayez pitié de nous.}
\invocatio{Christe, audi nos.}{Jésus-Christ, écoutez-nous.}
\invocatio{Christe, exáudi nos.}{Jésus-Christ, exaucez-nous.}
\rinvocatio{Pater de cælis, Deus,}{Père céleste qui êtes Dieu,}%
{miserére nobis.}{ayez pitié de nous.}
\invocatio{Fili, Redémptor mundi, Deus,}{Fils Rédempteur du monde, qui êtes Dieu,}
\invocatio{Spíritus Sancte, Deus,}{Esprit Saint, qui êtes Dieu,}
\invocatio{Sancta Trínitas, unus Deus,}{Trinité Sainte, qui êtes un seul Dieu,}
\invocatio{Cor Iesu, Fílii Patris ætérni,}{Cœur de Jésus, Fils du Père éternel,}
\invocatio{Cor Iesu, in sinu Vírginis Matris a Spíritu Sancto formátum,}{Cœur de Jésus, formé par le Saint-Esprit, dans le sein de la Vierge Mère,}
\invocatio{Cor Iesu, Verbo Dei substantiáliter unítum,}{Cœur de Jésus, uni substantiellement au Verbe de Dieu,}
\invocatio{Cor Iesu, maiestátis infinítæ,}{Cœur de Jésus, d’une infinie majesté,}
\invocatio{Cor Iesu, templum Dei sanctum,}{Cœur de Jésus, temple saint de Dieu,}
\invocatio{Cor Iesu, tabernáculum Altíssimi,}{Cœur de Jésus, tabernacle du Très-Haut,}
\invocatio{Cor Iesu, domus Dei et porta cæli,}{Cœur de Jésus, maison de Dieu et porte du ciel,}
\invocatio{Cor Iesu, fornax ardens caritátis,}{Cœur de Jésus, fournaise ardente de charité,}
\invocatio{Cor Iesu, iustítiæ et amóris receptáculum,}{Cœur de Jésus, sanctuaire de la justice et de l’amour,}
\invocatio{Cor Iesu, bonitáte et amóre plenum,}{Cœur de Jésus, plein d’amour et de bonté,}
%
\invocatio{Cor Iesu, virtútum ómnium abýssus,}{Cœur de Jésus, abîme de toutes les vertus,}
\invocatio{Cor Iesu, omni laude digníssimum,}{Cœur de Jésus, très digne de toute louange,}
\invocatio{Cor Iesu, rex et centrum ómnium córdium,}{Cœur de Jésus, roi et centre de tous les cœurs,}
\invocatio{Cor Iesu, in quo sunt  omnes thesáuri sapiéntiæ et sciéntiæ,}{Cœur de Jésus, en qui sont tous les trésors de la sagesse et de la science,}
\invocatio{Cor Iesu, in quo hábitat omnis plenitúdo divinitátis,}{Cœur de Jésus, en qui réside toute la plénitude de la divinité,}
\invocatio{Cor Iesu, in quo Pater sibi bene complácuit,}{Cœur de Jésus, en qui le Père céleste a mis toutes ses complaisances,}
\invocatio{Cor Iesu, de cuius plenitúdine omnes nos accépimus,}{Cœur de Jésus, dont la plénitude se répand sur chacun de nous,}
\invocatio{Cor Iesu, desidérium cóllium æternórum,}{Cœur de Jésus, le désiré des collines éternelles,}
\invocatio{Cor Iesu, pátiens et multæ misericórdiæ,}{Cœur de Jésus, patient et très miséricordieux,}
%
\invocatio{Cor Iesu, dives in omnes qui ínvocant te,}{Cœur de Jésus,  libéral envers tous ceux qui vous invoquent,}
\invocatio{Cor Iesu, fons vitæ et sanctitátis,}{Cœur de Jésus, source de vie et de sainteté,}
\invocatio{Cor Iesu, propitiátio pro peccátis nostris,}{Cœur de Jésus, propitiation pour nos péchés,}
\invocatio{Cor Iesu, saturátum oppróbriis,}{Cœur de Jésus, rassasié d’opprobres,}
\invocatio{Cor Iesu, attrítum propter scélera nostra,}{Cœur de Jésus, brisé de douleur à cause de nos péchés,}
\invocatio{Cor Iesu, usque ad mortem obédiens factum,}{Cœur de Jésus, obéissant jusqu’à la mort,}
\invocatio{Cor Iesu, láncea perforátum,}{Cœur de Jésus, percé par la lance,}
\invocatio{Cor Iesu, fons totíus consolatiónis,}{Cœur de Jésus, source de toute consolation,}
\invocatio{Cor Iesu, vita et resurréctio nostra,}{Cœur de Jésus, notre vie et notre résurrection,}
\invocatio{Cor Iesu, pax et reconciliátio nostra,}{Cœur de Jésus, notre paix et notre réconciliation,}
\invocatio{Cor Iesu, víctima peccatórum,}{Cœur de Jésus, victime des pécheurs,}
\invocatio{Cor Iesu, salus in te sperántium,}{Cœur de Jésus, salut de ceux qui espèrent en vous,}
\invocatio{Cor Iesu, spes in te moriéntium,}{Cœur de Jésus, espérance de ceux qui meurent dans votre amour,}
\invocatio{Cor Iesu, delíciæ Sanctórum ómnium,}{Cœur de Jésus, délices de tous les saints,}
%
%\invocatio{Agnus Dei, qui tollis peccáta mundi,}{Agneau de Dieu, qui effacez les péchés du monde,}
%\rinvocatio{parce nobis, Dómine.}{pardonnez-nous, Seigneur.}
\rinvocatio{Agnus Dei, qui tollis peccáta mundi,}{Agneau de Dieu, qui effacez les péchés du monde,}%
{parce nobis, Dómine.}{pardonnez-nous, Seigneur.}
\rinvocatio{Agnus Dei, qui tollis peccáta mundi,}{Agneau de Dieu, qui effacez les péchés du monde,}%
{exáudi nos, Dómine.}{exaucez-nous, Seigneur.}
\rinvocatio{Agnus Dei, qui tollis peccáta mundi,}{Agneau de Dieu, qui effacez les péchés du monde,}%
{miserére nobis.}{ayez pitié de nous.}
\end{litaniae}

\medskip
\versiculus{Iesu, mitis et húmilis Corde.}{Jésus, doux et humble de Cœur.\hspace{-1ex}}
\responsum{Fac cor nostrum secúndum Cor tuum.}{Rendez notre cœur semblable au vôtre.}

\oratio{%
Omnípotens sempitérne Deus, réspice in Cor dilectíssimi Fílii tui, et in laudes et satisfactiónes, quas in nómine peccatórum tibi persólvit,~† iísque misericórdiam tuam peténtibus, tu véniam concéde placátus,~\* in nómine eiúsdem Fílii tui Iesu Christi, qui tecum vivit et regnat in sǽcula sæculórum.%
}{%
Dieu tout-puissant et éternel, regardez le Cœur de votre Fils bien aimé et soyez attentif aux louanges et aux satisfactions, qu’il vous offre au nom des pécheurs. À ceux qui implorent votre miséricorde, accordez avec bienveillance le pardon, au nom de ce même Jésus-Christ, votre Fils, qui vit et règne avec vous dans tous les siècles des siècles.}


\titre{Parle, commande, règne}

%TODO:Partition:Parle, commande, règne:\lilypondfile[staffsize=15]{ly/ParleCommandeRegne/ParleCommandeRegne.ly}

\chanson[position=2col,numero=1]{ly/ParleCommandeRegne/ParleCommandeRegne}


\titre{Je suis chrétien}

%TODO:Partition:Je suis chrétien:\lilypondfile[staffsize=15]{ly/JeSuisChretien/JeSuisChretien.ly}

\chanson[position=2col,numero=1]{ly/JeSuisChretien/JeSuisChretien}


\titre{Invocations au Sacré-Cœur}

%TODO:Partition:Invocations au Sacré-Cœur:\lilypondfile[staffsize=15]{ly/InvocationsAuSacreCoeur/InvocationsAuSacreCoeur.ly}

\chanson[position=2col,numero=1]{ly/InvocationsAuSacreCoeur/InvocationsAuSacreCoeur}


\addchap{Chants à la Vierge}


\titre{Ave Maria de Lourdes}

\lilypondfile[staffsize=15]{ly/AveMariaDeLourdes/AveMariaDeLourdes.ly}

\chanson[position=2col,numero=2,premiercouplet=2,refrain=non]{ly/AveMariaDeLourdes/AveMariaDeLourdes}


\titre[espace=5]{Salve Mater}

\cantus{Autre}{SalveMater}{}{5.}

\chanson[position=2col,numero=2,premiercouplet=2]{ly/SalveMater/SalveMater}


\titre{Ave Maris Stella}

\cantus{Hymne}{AveMarisStella}{Hymn.}{1.}

\vspace{-1\baselineskip}
\chanson[position=2col,numero=2,premiercouplet=2]{ly/AveMarisStella/AveMarisStella}


\titre[table=non,espace=0]{Autre mélodie}

\cantus{Hymne}{AveMarisStella-OfficiumParvum}{Hymn.}{1.}

\pagebreak[3]


\needspace{6\baselineskip}
\titre{Je mets ma confiance}

%TODO:Partition:Je mets ma confiance:\lilypondfile[staffsize=15]{ly/JeMetsMaConfiance/JeMetsMaConfiance.ly}

\chanson[position=2col,numero=1]{ly/JeMetsMaConfiance/JeMetsMaConfiance}


\titre{Laudate Mariam}

%TODO:Partition:Laudate Mariam:\lilypondfile[staffsize=15]{ly/LaudateMariam/LaudateMariam.ly}

\chanson[position=2col,numero=1]{ly/LaudateMariam/LaudateMariam}


\titre{J'irai la voir un jour}

%TODO:Partition:J'irai la voir un jour:\lilypondfile[staffsize=15]{ly/JIraiLaVoirUnJour/JIraiLaVoirUnJour.ly}

\chanson[position=2col,numero=1]{ly/JIraiLaVoirUnJour/JIraiLaVoirUnJour}


\titre{Vierge sainte}

%TODO:Partition:Vierge sainte:\lilypondfile[staffsize=15]{ly/ViergeSainte/ViergeSainte.ly}

\chanson[position=2col,numero=1]{ly/ViergeSainte/ViergeSainte}


\titre{Chez nous, soyez Reine}

%TODO:Partition:Chez nous, soyez Reine:\lilypondfile[staffsize=15]{ly/ChezNousSoyezReine/ChezNousSoyezReine.ly}

\chanson[position=2col,numero=1]{ly/ChezNousSoyezReine/ChezNousSoyezReine}


\titre{Reine de France}

%TODO:Partition:Reine de France:\lilypondfile[staffsize=15]{ly/ReineDeFrance/ReineDeFrance.ly}

\chanson[numero=1]{ly/ReineDeFrance/ReineDeFrance}


\titre[espace=15]{Nous voulons Dieu}

%TODO:Partition:Nous voulons Dieu:\lilypondfile[staffsize=15]{ly/NousVoulonsDieu/NousVoulonsDieu.ly}

\chanson[position=2col,numero=1]{ly/NousVoulonsDieu/NousVoulonsDieu}


%\titre{Ô  ma reine, ô Vierge Marie}

%%TODO:Partition:Ô ma reine, ô Vierge Marie:\lilypondfile[staffsize=15]{ly/OMaReine/OMaReine.ly}

%\chanson[position=2col,numero=1]{ly/OMaReine/OMaReine}


\addchap{Chemin de Croix}


%\titre[table=Suivons sur la montagne sainte]{\hfill Suivons sur la montagne sainte\hfill}

%%TODO:Partition:Suivons sur la montagne sainte:\lilypondfile[staffsize=15]{ly/SuivonsSurLaMontagneSainte/SuivonsSurLaMontagneSainte.ly}

%\chanson[numero=1]{ly/SuivonsSurLaMontagneSainte/SuivonsSurLaMontagneSainte}


%\titre{Vive Jésus, vive sa Croix}

%%TODO:Partition:Vive Jésus, vive sa Croix:\lilypondfile[staffsize=15]{ly/ViveJesusViveSaCroix/ViveJesusViveSaCroix.ly}

%\chanson[position=2col,numero=1]{ly/ViveJesusViveSaCroix/ViveJesusViveSaCroix}


\titre{Stabat Mater}

\cantus{Hymne}{StabatMater}{Hymn.}{6.}

\chanson[position=2col,numero=2,premiercouplet=2]{ly/StabatMater/StabatMater}


\titre[espace=3]{Jésus-Christ monte au Calvaire}

{\widowpenalties 3 0 0 0 \clubpenalties 3 0 0 0
\lilypondfile[staffsize=15]{ly/JesusChristMonteAuCalvaire/JesusChristMonteAuCalvaire.ly}
}

\chanson[position=2col,numero=2,premiercouplet=2]{ly/JesusChristMonteAuCalvaire/JesusChristMonteAuCalvaire}


\titre{Conclusion du Chemin de Croix}

{\footnotesize
\versio{%
℣. Adorámus te, Christe, et benedícimus tibi.%
}{%
℣. Nous vous adorons, ô Christ, et nous vous bénissons.%
}

\versio{%
℟. Quia per sanctam Crucem tuam redemísti mundum.%
}{%
℟. Parce que vous avez racheté le monde par votre sainte Croix.%
}

\versio{%
℣. Ora pro nobis, Virgo dolorosíssima.%
}{%
℣. Priez pour nous, Vierge des douleurs.%
}

\versio{%
℟. Ut digni efficiámur promissiónibus Christi.%
}{%
℟. Afin que nous soyons dignes des promesses de Notre Seigneur Jésus-Christ.%
}

\versio{%
℣. Signásti, Dómine, servum tuum Francíscum.%
}{%
℣. Vous avez marqué, Seigneur, votre serviteur saint François.%
}

\versio{%
℟. Signis Redemptiónis nostræ.%
}{%
℟. Des marques de notre Rédemption.%
}

\versio{%
℣. Orémus pro Pontífice nostro Francísco.%
}{%
℣. Prions pour notre Pontife François.%
}

\versio{%
R. Dóminus consérvet eum et vivíficet eum, et beátum fáciat eum in terra,
et non tradat eum in ánimam inimicórum eius.%
}{%
℟. Que le Seigneur le protège et le vivifie, qu'il le rende heureux sur la terre,
et qu'il ne le livre pas à la merci de ses ennemis.%
}

\versio{%
℣. Orémus pro fidélibus defúnctis.%
}{%
℣. Prions pour les fidèles défunts.%
}

\versio{%
℟. Réquiem ætérnam dona eis, Dómine, et lux perpétua lúceat eis.%
}{%
℟. Donnez-leur le repos éternel, Seigneur, et que la lumière sans fin brille
sur eux.%
}

}

\versio{%
Orémus.%
}{%
Prions.%
}

\rubrica{Suivent six oraisons, à la fin desquelles on chante :}


\titre[espace=3]{Parce, Domine}

\cantus{Antienne}{ParceDomine-ref}{Ant.}{1.}

%\titre{Ô Croix dressée sur le monde}

%%TODO:Partition:Ô Croix dressée sur le monde:\lilypondfile[staffsize=15]{ly/OCroixDresseeSurLeMonde/OCroixDresseeSurLeMonde.ly}

%\chanson[position=2col,numero=1]{ly/OCroixDresseeSurLeMonde/OCroixDresseeSurLeMonde}


\addchap{Salut du Saint-Sacrement}


\titre{Adoro te}

\cantus{Hymne}{AdoroTe}{Hymn.}{5.}

\chanson[position=2col,numero=2,premiercouplet=2]{ly/AdoroTe/AdoroTe}


\titre{Ubi caritas}

\cantus{Antienne}{UbiCaritas}{Ant.}{6.}


\titre{Pange lingua}

\cantus{Hymne}{PangeLingua_Corporis-sansTantum}{Hymn.}{3.}


\titre{Prière pour le Pape}

\cantus{Antienne}{TuEsPastor}{Ant.}{1.}

\versio{%
℣. Tu es Petrus.%
}{%
℣. Tu es Pierre.%
}

\versio{%
℟. Et super hanc petram ædificábo Ecclésiam meam.%
}{%
℟. Et sur cette pierre j'édifierai mon Église.%
}


\titre[espace=3]{Tantum ergo}

\cantus{Hymne}{TantumErgo}{Hymn.}{3.}

\versio{%
℣. Panem de cælo præstitísti eis.

℟. Omne delectaméntum in se habéntem.%
}{%
℣. Vous leur avez donné le pain du ciel.

℟. Qui renferme en lui tous les délices.%
}


\titre{O Salutaris Hostia}

\lilypondfile[staffsize=15]{ly/OSalutarisHostia/OSalutarisHostia.ly}


\titre{Loué soit à tout instant}

\lilypondfile[staffsize=15]{ly/LoueSoitAToutInstant/LoueSoitAToutInstant.ly}

\chanson[position=2col,numero=2,premiercouplet=2,refrain=non]{ly/LoueSoitAToutInstant/LoueSoitAToutInstant}


\titre{Lauda Ierusalem}

\lilypondfile[staffsize=15]{ly/LaudaIerusalem/LaudaIerusalem.ly}

\psalmus[primus=3,numerus=2]{147}

\gloria


\titre{Lauda Sion}

\lilypondfile[staffsize=15]{ly/LaudaSion/LaudaSion.ly}

\chanson[position=2col,premiercouplet=2,numero=2,refrain=non]{ly/LaudaSion/LaudaSion}


\titre{Benedictus}

\lilypondfile[staffsize=15]{ly/Benedictus/Benedictus.ly}

\psalmus[tonus=6F,primus=2]{Benedictus}


\titre[table=Oh l'auguste Sacrement]{Oh ! l'auguste Sacrement}

%TODO:Partition:Oh ! l'auguste Sacrement:\lilypondfile[staffsize=15]{ly/OhLAugusteSacrement/OhLAugusteSacrement.ly}

\chanson[position=2col,numero=1]{ly/OhLAugusteSacrement/OhLAugusteSacrement}


\titre{Sur la patène}

%TODO:Partition:Sur la patène:\lilypondfile[staffsize=15]{ly/SurLaPatene/SurLaPatene.ly}

\chanson[position=2col,numero=1]{ly/SurLaPatene/SurLaPatene}


\addchap{Chant à S. Pie X}
\index{Sancte Pie X}

\lilypondfile[staffsize=15]{ly/SanctePieX/SanctePieX.ly}



%%%%%%%%%%%%%%%%%%%%%%%%%%%%%%%%%%%%%%%%%%%%%%%%%%%%%%%%%%%%%%%%%%%%%%%%%%%%%%%%

\tableofcontents

\vfill\thispagestyle{empty}
\printindex

\cleardoublepage

\addchap{Plan des sanctuaires}

\includegraphics[width=\textwidth]{img/Plan}

\vfill
{\centering\parbox{.6\textwidth}{%
\begin{enumerate}
\item Grotte des apparitions ;
\item entrée de la basilique Saint-Pie X ;
\item église Sainte-Bernadette ;
\item accueil Notre-Dame ;
\item porte des lacets.
\end{enumerate}%
}\par
}
\vfill
\clearpage

{\thispagestyle{empty}\centering
\vspace*{\stretch{2}}

\includegraphics[width=.3\textwidth]{img/fsspx}

\vspace*{\stretch{3}}

\footnotesize 11, rue Cluseret − 92280 SURESNES CEDEX\par}


\end{document}
%Apercu:evince 00-Document.pdf:
%Cible::
%Esclaves:./esclaves.sh:
